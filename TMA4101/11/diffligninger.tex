
% Morten -- jeg skriver bare en veldig plain tex-fil her, så kan du gjøre det du vil med innholdet etterpå.


\documentclass{article}
\usepackage[utf8]{inputenc}
\usepackage[a4paper]{geometry}
\usepackage[norsk]{babel}
\usepackage{amssymb}
\usepackage{amsmath}
\usepackage{amsthm}
\usepackage{multicol}
\usepackage{scrextend}
\usepackage{mdframed}
\usepackage{hyperref}
\usepackage{listings}
\usepackage{tikz}
\usetikzlibrary{patterns,decorations.pathmorphing}
\usepackage{pgfplots}
\usepackage{array}
\usepackage{mdframed}
\usepackage[american]{circuitikz}
\usepackage[small]{titlesec}

\hypersetup{colorlinks=true, 
    linkcolor=blue, 
    filecolor=magneta,
    urlcolor=cyan,
    citecolor=black,
}

\theoremstyle{plain}
\newtheorem{teorem}{Teorem}\surroundwithmdframed{teorem}

\theoremstyle{definition}
\newtheorem{definisjon}[teorem]{Definisjon}
\newtheorem{eksempel}[teorem]{Eksempel}
\newtheorem{funfact}[teorem]{Funfact}

\theoremstyle{remark}
\newtheorem{merknad}[teorem]{Merknad}
\newtheorem*{bevis}{Bevis}

\newtheorem{oppg}{}
\newtheorem{innercustomoppg}{}
\newenvironment{customoppg}[1]
{\renewcommand\theinnercustomoppg{#1}\innercustomoppg}
{\endinnercustomoppg}

% Endre på disse etter hvilken notasjon du vil ha.
\newcommand{\diff}[1]{\mathop{d#1}} 
\newcommand{\fcn}{x}
\newcommand{\boldvec}[1]{\boldsymbol{\mathrm{#1}}}
\newcommand{\expfcn}[1]{e^{#1}}
\newcommand{\bigabs}[1]{\big|#1\big|}
\newcommand{\biggabs}[1]{\bigg|#1\bigg|}
\newcommand{\bigparanth}[1]{\big(#1\big)}
\newcommand{\biggparanth}[1]{\bigg(#1\bigg)}
\newcommand{\bigbrac}[1]{\big[#1\big]}
\newcommand{\biggbrac}[1]{\bigg[#1\bigg]}
\newcommand{\imagunit}{\mathrm{j}}

\DeclareMathOperator{\imaginary}{Im}
\DeclareMathOperator{\real}{Re}


\title{Førsteordens differensialligninger}
\author{}
\date{}

\begin{document}
\maketitle

Dette kapitlet er en introduksjon til førsteordens ordinære differensialligninger. Vi skal se på både lineære og ikkelineære ligninger. For lineære differensialligninger finnes det god teori på eksistens og unikhet av løsninger, flere universelle løsningsteknikker, og vi kan ofte finne analytiske løsninger for disse. Dette gjelder i mye mindre grad for ikkelineære ligninger. For disse er det mye vanskeligere å utlede analytiske løsninger, og vi er avhengig av kvalitative eller numeriske metoder. 

Vi begynner med å se på noen eksempler hvor differensialligninger oppstår som modeller for fysiske situasjoner. Vi skal komme tilbake til hvordan vi kan løse disse mot slutten av dette notatet.

\begin{eksempel} \label{eks:krets_1}
    Følgende figur viser en (RL-)krets med en spenningskilde, en motstand og en spole.
    \begin{center}
        \begin{circuitikz}
          \draw
          (0,0)
          to[american voltage source, voltage dir=old, l={$V_s$}, i^={$I(t)$}] (0,3)
          to[european resistor, l_ = $R$] (3,3)
          to (3,0)
          to[cute inductor, l_ = $L$] (0, 0);
        \end{circuitikz}
      \end{center}
      Ved Ohms lov er spenningsfallet over motstanden lik $RI$ og spenningsfallet over spolen er $L\dot{I}$. Kirchhoffs spenningslov, som sier at summen av potensialforskjellene over en lukket strømsløyfe må utligne hverandre, gir dermed at
      \begin{equation*}
          \dot{I} + \frac{R}{L} I = \frac{V_s}{L}.
      \end{equation*}
      Dette er et eksempel på en førsteordens lineær differensialligning, hvor den ukjente er funksjonen $I(t)$.
\end{eksempel}

\begin{eksempel} \label{eks:newtons_kj_lov_1}
    Newtons kjølelov sier at endringen i temperaturen til et objekt er direkte proporsjonal med temperaturforskjellen mellom objektet og omgivelsene. Dersom Sindre skal steke ribbe til julaften kan temperaturen til ribben etter at det er satt inn i ovnen bestemmes ved å løse
    \begin{equation*}
        \dot{T} = k(T - T_0).
    \end{equation*}
    Proporsjonalitetskonstanten $k$ for systemet må riktignok bestemmes først, men dette kan Sindre selv gjøre på en eller annen måte.
\end{eksempel}

\begin{eksempel} \label{eks:populasjon_1}
    I studet av populasjonsvekst møter man ofte den logistiske ligningen
    \begin{equation*}
        \dot{x}(t) = A \fcn(t) - B (\fcn(t))^2,
    \end{equation*}
    hvor $\fcn(t)$ beskriver en populasjonsstørrelse som funksjon av tid. Dersom vi setter $B = 0$ får vi en malthusiansk modell, etter den britiske samfunnsøkonomen Thomas Robert Malthus, som gir eksponensiell populasjonsvekst. Senere skal vi se at leddet $-B x^2$ begrenser veksten i populasjonsmodellen.
\end{eksempel}


\section*{En kort oversikt}

En førsteordens differensialligning er en relasjon mellom en funksjon og dens deriverte. Vi skal her jobbe utelukkende med ordinære differensialligninger, altså ligninger for funksjoner med bare én ukjent. Generelt kan vi skrive
\begin{equation} \label{eq:generell_ekspl_forsteordens}
    \dot{\fcn}(t) = f(t, \fcn(t)),
\end{equation}
hvor $\dot{\fcn}$ er den deriverte av funksjonen $\fcn$ med hensyn på variabelen $t$, og $f$ er en funksjon. Dersom $f$ ikke eksplisitt er avhengig av $t$, altså dersom vi kan skrive $f = f(\fcn(t))$, kaller vi ligningen autonom. Videre, dersom $f$ er en lineær funksjon i $\fcn$ er differensialligningen lineær.

Vi sier at en funksjon $\fcn(t)$ løser ligning \eqref{eq:generell_ekspl_forsteordens} dersom den er kontinuerlig deriverbar på et åpent intervall $I = (a, b)$ og tilfredsstiller ligningen på dette intervallet. 

Dersom vi kan finne et eksakt uttrykk ved hjelp av et endelig antall kjente funksjoner (polynomer, trigonometriske funksjoner, eksponensialfunksjoner, osv.) for løsningen til en differensialligning kalles dette ofte en analytisk løsning til ligningen. Dette står i kontrast til numeriske løsninger, som er tilnærminger.

En differensialligning kommer sjelden uten en historie som vi må ta hensyn til dersom vi skal forstå den. Initialverdiproblemer er et eksempel på dette, hvor differensialligning har en tilhørende initialbetingelse,
\begin{equation} \label{eq:ivp}
    \left\{
        \begin{aligned}
            & \dot{x} = f(\fcn, t), \\
            & x(t_0) = x_0.
        \end{aligned}
    \right.
\end{equation}

La oss nå vri å vende litt på initialverdiproblemet gitt i \eqref{eq:ivp}. Husk at derivasjon er definert som en grenseverdi av stigningen til en funksjon. På denne måten er ligning \eqref{eq:generell_ekspl_forsteordens} grensen til ligningen
\begin{equation*}
    \frac{\fcn(t + h) - \fcn(t)}{h} = f(\fcn(t), t),
\end{equation*}
med $h \rightarrow 0$. På den annen side kan vi integrere og bruke analysens fundamentalterem til å skrive ligningen på integralform, nemlig
\begin{equation} \label{eq:integralform}
    \fcn(t) = \fcn_0 + \int_{t_0}^{t} f(\fcn(\tau), \tau) \diff{\tau}.
\end{equation}
Til integralet på høyre side kan vi bruke numeriske approksimasjoner, som gir opphav til numeriske løsningsteknikker for problemet. En annen idé er å prøve fikspunktiterasjon på \eqref{eq:integralform}. Dette gir kanskje en smakebit på at differensiallignniger kan ses på i forskjellige lys, og at det er en stor variasjon av teknikker og triks tilgjengelig.


\section*{Lineære ligninger}

En førsteordens lineær differensialligning kan skrives på standard form som
\begin{equation} \label{eq:linear_forsteordens}
    \dot{\fcn}(t) + p(t) \fcn(t) = q(t).
\end{equation}
Dersom $r(t) = 0$ sier vi at ligningen er homogen.

Vi viser nå hvordan den inhomogene ligningen \eqref{eq:linear_forsteordens} kan løses generelt. Multipliser me den funksjon $\expfcn{P(t)}$ på begge sider,
\begin{equation*}
    \dot{\fcn}(t) \expfcn{P(t)} + \fcn(t) p(t) \expfcn{P(t)} = q(t) \expfcn{P(t)},
\end{equation*}
hvor $P(t)$ er en antiderivert til $p(t)$. Ved kjerneregelen for derivasjon er venstre side nå lik den deriverte av $\fcn(t) \expfcn{P(t)}$. Integrerer vi ligningen med hensyn på $t$ får vi dermed
\begin{equation*}
    \fcn(t) \expfcn{P(t)} = \int q(t) \expfcn{P(t)} \diff{t} + C, 
\end{equation*}
hvor $C$ er en vilkårlig integrasjonskonstant. Løsningen til ligningen er dermed
\begin{equation} \label{eq:generell_losn_linear}
    \fcn(t) = \expfcn{-P(t)} \biggparanth{\int q(t) \expfcn{P(t)} \diff{t} + C}, \qquad P(t) = \int p(t) \diff{t}.
\end{equation}

\begin{eksempel}
    Vi vet at ligningen
    \begin{equation*}
        \dot{\fcn}(t) + a \fcn(t) = 0
    \end{equation*}
    har løsning
    \begin{equation*}
        \fcn(t) = \expfcn{-a t}.
    \end{equation*}
    Dette stemmer overens med \eqref{eq:generell_losn_linear}.
\end{eksempel}

\begin{eksempel}
    \begin{equation*}
        \dot{\fcn}(t) - 2\fcn(t) = 3\expfcn{t}.
    \end{equation*}
    \begin{equation*}
        \dot{\fcn}(t) \expfcn{-2t} - 2\fcn(t) \expfcn{-2t} = 3\expfcn{t} \expfcn{-2t}
    \end{equation*}
    \begin{equation*}
        \frac{\diff{}}{\diff{t}} \bigparanth{\fcn(t) \expfcn{-2t}} = 3\expfcn{t} \expfcn{-2t}
    \end{equation*}
    Ved å integrere får man
    \begin{equation*}
        x(t) = \expfcn{2t} \biggparanth{}
    \end{equation*}
\end{eksempel}


Observasjon om homogen/inhomogen løsning med ubestemt konstant.

\begin{teorem}
    Eksistens og unikhet.
\end{teorem}


Dersom funksjonen som inngår i ligningen er en vektorfunksjon med $n$ komponenter
\begin{equation*}
    \boldsymbol{x}(t) =
    \begin{pmatrix}
        x_1(t) \\
        x_2(t) \\
        \vdots \\
        x_n(t)
    \end{pmatrix}
\end{equation*}
har vi et ligningssystem av dimensjon $n$, helt i samsvar med det vi har sett tidligere i lineær algebra. Dett blir viktig senere, men vi skal her nøye oss med skalare funksjoner $\fcn(t)$ og dens deriverte.


\section*{Ikkelineære ligninger}

Ofte svært vanskelig finne formler for, løse analytisk.

Separable ligninger.

Fasediagram i en dimensjon. Motiverende eksempel.

Tredje metode: numerikk.


\section*{Numeriske metoder}

Intro, arbeidshest.

Euler ekslisitt, implisitt.

Trapesmetoden.

Heuns metode.

Midtpunktmetoden.


\section*{Oppgaver}

\begin{oppg}
    Picarditerasjon.
\end{oppg}

\end{document}
