\documentclass[a4paper,norsk,11pt]{interaktiv}

% Importerte pakker
\usepackage{float}
\restylefloat{figure}
\usepackage{ifluatex}
\usepackage{subfigure}
\usepackage{tikz}
\usetikzlibrary{arrows}
\usepackage[parfill]{parskip}    		% Activate to begin paragraphs with an empty line rather than an indent
\usepackage{graphicx}


\ifluatex
  \usepackage{fontspec}
  \setmainfont{Calibri}
  \usepackage{unicode-math}
  \setmathfont{Cambria Math}
\else
  \usepackage[utf8]{inputenc}
\fi

% Underfigurer
\renewcommand{\thesubfigure}{(\arabic{subfigure})}

% Overskrift
\emnekode{TMA4101}
\emnenavn{Matematikk 1 for MTELSYS}
\title{Øving 3 - Følger, rekker og numeriske likningsløsere II}
\author{8. september kl 1400}


% Nye kommandoer
\newcommand{\dee}{\mathop{}\!{d}}

\begin{document}
\pagenumbering{gobble}

\maketitle



\section*{Obligatoriske oppgaver}


\begin{oppgave}{E1}
Vis at rekken
\[
\sum_{n=1}^\infty \frac{(-3x)^n}{n}
\]
konvergerer dersom $-\frac{1}{3} < x \leq \frac{1}{3}$.
\end{oppgave}


\begin{oppgave}{E2}
Lag en pythonfunksjon som tar inn $x$ og $N$, 
og evaluerer den $N$-te partialsummen til rekken
\[
f(x)=2\sum_{n=1}^\infty \frac{(-1)^{n+1}}{n}\sin nx.
\]
Kjør koden din for forskjellige verdier av $x \in [-\pi,\pi]$, og $N=100000$. 
Kan du si noe om $f$?
\end{oppgave}




\section*{Anbefalte oppgaver}

  Avgjør om følgende rekker konvergerer eller divergerer.


\begin{oppgave}{C1}
    $\quad\displaystyle \sum_{n = 1}^\infty \frac{\sin n}{n^2 + 1}$
\end{oppgave}

\begin{oppgave}{B2}
    $\quad\displaystyle \sum_{n = 0}^\infty \left(\frac\pi2 - \arctan
      n\right)$
\end{oppgave}
\begin{oppgave}{D3}
    $\quad\displaystyle \sum_{n = 0}^\infty \frac{(-1)^n (n^2 - 1)}{n^2 +1}$
\end{oppgave}
\begin{oppgave}{D4}
    $\quad\displaystyle \sum_{n = 1}^\infty \frac{(-1)^n}{\sqrt n}$
\end{oppgave}
\begin{oppgave}{C5}
    $\quad\displaystyle \sum_{n=1}^{\infty} \sqrt{n+1}-\sqrt{n}$
\end{oppgave}

\section*{Relevante eksamensoppgaver fra TMA4100}

\begin{oppgave}{C}
2019H oppgave 9
\end{oppgave}

\begin{oppgave}{B}
2019K oppgave 7
\end{oppgave}

\begin{oppgave}{B}
2018H oppgave 7
\end{oppgave}

\begin{oppgave}{C}
2018K oppgave 6
\end{oppgave}

\begin{oppgave}{B}
2017H oppgave 6
\end{oppgave}

\begin{oppgave}{A}
2001H oppgave 5
\end{oppgave}

\begin{oppgave}{C}
2001K oppgave 5
\end{oppgave}

\begin{oppgave}{C}
1999H oppgave 1
\end{oppgave}

\begin{oppgave}{B}
1999H oppgave 4
\end{oppgave}

\begin{oppgave}{B}
1999H oppgave 7 \\
(Hint: konvergensradien i obligatorisk oppgave E2 er $\frac{1}{3}$.)
\end{oppgave}






\section*{Vanskelige oppgaver}

\begin{oppgave}{1}
Bevis sammenlikningstesten. 
\end{oppgave}

\begin{oppgave}{2}
Bevis forholdstesten. (Hint: Bruk sammenlikningstesten og en passende geometrisk rekke.)
\end{oppgave}
%
%\begin{oppgave}{3}
%Følgende test kalles rot-testen. 
%La $f$ være en positive følge, og la
%\[
%\rho=\lim_{n\to \infty}\sqrt[n]{f(n)}.
%\]
%Dersom $\rho >1$ er $\sum f$ divergent. 
%Dersom $\rho <1$ er $\sum f$ konvergent. 
%Dersom $\rho =1$ gir testen ingen informasjon.
%
%Bevis rottesten. (Hint: Beviset er likner veldig på beviset for forholdstesten.)
%\end{oppgave}


%\begin{oppgave}{3}
%Vis at 
%\[
%\exp{(x+y)}=\exp{x}\exp{y}.
%\]
%\end{oppgave}

\end{document}
