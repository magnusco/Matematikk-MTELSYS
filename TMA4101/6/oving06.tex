\documentclass[a4paper,norsk,11pt]{interaktiv}

% Importerte pakker
\usepackage{float}
\restylefloat{figure}
\usepackage{ifluatex}
\usepackage{subfigure}
\usepackage{tikz}
\usetikzlibrary{arrows}
\usepackage[parfill]{parskip}    		% Activate to begin paragraphs with an empty line rather than an indent
\usepackage{graphicx}
\usepackage{systeme}
\usepackage[european,americanvoltages]{circuitikz}

\newenvironment{amatrix}[1]{% "augmented matrix"
  \left[\begin{array}{*{#1}{c}|c}
}{%
  \end{array}\right]
}

\newcommand{\V}[1]{\mathbf{#1}}
\newcommand{\vv}[2]{\begin{bmatrix} #1 \\ #2 \end{bmatrix}}
\newcommand{\vvS}[2]{\left[ \begin{smallmatrix} #1 \\ #2 \end{smallmatrix} \right]}
\newcommand{\vvv}[3]{\begin{bmatrix} #1 \\ #2 \\ #3 \end{bmatrix}}
\newcommand{\vvvv}[4]{\begin{bmatrix} #1 \\ #2 \\ #3 \\ #4 \end{bmatrix}}
\newcommand{\vvvvv}[5]{\begin{bmatrix} #1 \\ #2 \\ #3 \\ #4 \\ #5 \end{bmatrix}}
\newcommand{\vn}[2]{\vvvv{#1_1}{#1_2}{\vdots}{#1_#2}}
\newcommand{\N}{\mathbb{N}}
\newcommand{\Z}{\mathbb{Z}}
\newcommand{\Q}{\mathbb{Q}}
\newcommand{\R}{\mathbb{R}}
\newcommand{\C}{\mathbb{C}}

\newcommand{\M}{\mathcal{M}} % vektorrom av matriser
\newcommand{\Cf}{\mathcal{C}} % vektorrom av kontinuerlige funksjoner
\renewcommand{\P}{\mathcal{P}} % vektorrom av polynomer
\newcommand{\B}{\mathscr{B}} % basis
\newcommand{\roweq}{\sim}
\DeclareMathOperator{\Sp}{Sp}
\DeclareMathOperator{\Null}{Null}
\DeclareMathOperator{\Col}{Col}
\DeclareMathOperator{\Row}{Row}
\DeclareMathOperator{\rank}{rank}
\DeclareMathOperator{\im}{im}
\DeclareMathOperator{\id}{id}
\DeclareMathOperator{\Hom}{Hom}
\newcommand{\tr}{^\top}
\newcommand{\koord}[2]{[\,{#1}\,]_{#2}} % koordinater mhp basis

\ifluatex
  \usepackage{fontspec}
  \setmainfont{Calibri}
  \usepackage{unicode-math}
  \setmathfont{Cambria Math}
\else
  \usepackage[utf8]{inputenc}
\fi

% Underfigurer
\renewcommand{\thesubfigure}{(\arabic{subfigure})}

% Overskrift
\emnekode{TMA4101}
\emnenavn{Matematikk 1 for MTELSYS}
\title{Øving 6 - Lineær algebra}
%\author{Fourieranalyse}


% Nye kommandoer
\newcommand{\dee}{\mathop{}\!{d}}

\begin{document}
\pagenumbering{gobble}

\maketitle


\section*{Obligatoriske oppgaver}

\begin{oppgave}{E1}
Kretsen under modellerer en høyttaler, 
og du finner den i ERT-ykt 7.
\begin{center}
	\begin{circuitikz}
		\draw  (0,1) to [V=$V$] (0,4) to [ R=$R_2$,i=$i_1$] (4,4) to [R=$R_1$,i>_=$i_x$] (4,1) to [R=$R_2$]  (0,1); 
		\draw  (4,4) to [R=$R_2$,i>^=$i_2$] (8,4) to [R=$R_1$] (8,1) to [R=$R_2$] (3,1) ;
	\end{circuitikz}
\end{center}
\begin{itemize}
\item[a)] Skriv $i_x$ som en funksjon av $i_1$ og $i_2$. 
\item[b)] Bruk Kirchhoffs spenningslov
til å sette opp en $2\times 2$ lineært likningssystem for $i_1$ og $i_2$. 
Løs systemet, og sammenlikne med ERT-ykt 7. 
\end{itemize}
\end{oppgave}

\begin{oppgave}{E2}
La 
\[
A=
\begin{bmatrix}
1 & 1 & 1 \\ 2 & 1 & 0 \\ 4 & 2 & 1
\end{bmatrix}
\quad
\text{og }
\quad
b=
\begin{bmatrix}
0 \\ 0 \\ 1 
\end{bmatrix} 
\]
Finn $A^{-1}$ og løs $Ax=b$.
\end{oppgave}




\section*{Anbefalte oppgaver}



\begin{oppgave}{E1}
Skriv likningssystemet 
\[
\systeme{
x - 4y + 28z = -2,
-x + y - 7z = -31,
x + 2y - 14z = 64
}
\]
om til matriseform, og løs.
\end{oppgave}

\begin{oppgave}{D2}
	Hvilke av matrisene 	
		\[
		 \quad
		\begin{bmatrix}
		1 & 5 & 0 & 0 \\
		0 & 0 & 0 & 1
		\end{bmatrix},
		\quad
		\begin{bmatrix}
		1 & 0 \\
		0 & 1 \\
		0 & -1
		\end{bmatrix},
		\quad
		\begin{bmatrix}
		0 & 2 & 1 \\
		0 & 0 & 4 \\
		0 & 0 & 0
		\end{bmatrix},
		\quad \text{og} \quad		
		\begin{bmatrix}
		0 & 0 & 0 \\
		0 & 0 & 0 \\
		0 & 0 & 0
		\end{bmatrix}
		\]
	er på trappeform?  Hvilke av dem er på
	redusert trappeform?
\end{oppgave}


\begin{oppgave}{C3}
Løs ligningssystemene

\begin{itemize}
\item[a)]
$
\begin{amatrix}{3}
1 & 1 & -1 & 0 \\
  2 &- 3 & 1 & 1  \\
 -1 &  1 & 2 &  4
\end{amatrix}
$
\item[b)]
$
\begin{amatrix}{3}
  1 & 1 & -1 & 0 \\
 2 & - 3 & 1 & 1\\
 4 & -1 & -1 & 3
\end{amatrix}
$
\item[c)]
$
\begin{amatrix}{3}
		1 & 1 & -1 &  0 \\
		2 &- 3 & 1 & 1 \\
		4 &- 1 &- 1 & 1
\end{amatrix}
$
\item[d)]
$
\begin{amatrix}{3}
	1 & 1 &-  1 & 0 \\
		2 & - 3 &1&  1 
		% 2x + 6y + 12z  8
\end{amatrix}
$
\item[e)]
$
\begin{amatrix}{2}
i  & 1   & -1 \\
 1 &i  & i
\end{amatrix}
$
\item[f)]
$
\begin{amatrix}{2}
1-i  & 1   & 1 \\
 1 &i & 1+i
\end{amatrix}
$
\item[g)]
$
\begin{amatrix}{3}
0 & 1  & 1 & 1 \\
i & 1 & 1 & 1+i
\end{amatrix}
$
\item[i)]
$
\begin{amatrix}{3}
3 & -6  & 6 & -15 \\
1 & 1 & 4 & 10
\end{amatrix}
$
\end{itemize}

\end{oppgave}





\begin{oppgave}{C4}
Anta at vi har et ligningssystem med $m$~ligninger og~$n$ ukjente.
Hvilke av de ni forskjellige tilfellene i følgende tabell er mulige?
\[
\begin{array}{r|c|c|c|}
                                & m < n & m = n & m > n \\ \hline
\text{ingen løsninger}          &       &       &       \\ \hline
\text{én løsning}               &       &       &       \\ \hline
\text{uendelig mange løsninger} &       &       &       \\ \hline
\end{array}
\]
\end{oppgave}


\begin{oppgave}{C5}
La $z$ være en løsning av ligningen $z^2+z+1=0$. 
Finn en løsning av ligningssystemet med totalmatrise
\[
\begin{amatrix}{4}
1 & 1 & 1 & 3 & 9 \\
1 & 1& 1& -1& 1\\
1  & z& z^2& 0& 0\\
1     & z^2& z& 0 &0\\
\end{amatrix}
\]
\end{oppgave}


\begin{oppgave}{D6}
Avgjør hvorvidt ligningssystemet gitt ved
\[
\begin{amatrix}{3}
1 & 0 & 1 & 0 \\
1 & 2& 3&  5\\
1  & -2 & -1& 1\\
0     & -4& -1& -1\\
\end{amatrix}
\]
har en løsning.
\end{oppgave}




\begin{oppgave}{E7}
La $\V{u} = \vv{3}{2}$ og~$\V{v} = \vv{-1}{1}$ være to vektorer
i~$\R^2$.

Regn ut $\V{u} + \V{v}$ og $\frac{1}{2} \V{u} - 2 \V{v}$,
og tegn en figur som viser vektorene $\V{u}$, $\V{v}$, $\V{u} + \V{v}$ og
$\frac{1}{2} \V{u} - 2 \V{v}$ i planet.
\end{oppgave}


\begin{oppgave}{D8}
Finn ut om en vektor er en lineærkombinasjon av de andre:

a) \quad
$
\begin{bmatrix}
1\\
2
\end{bmatrix},
\begin{bmatrix}
2\\
3
\end{bmatrix},
\begin{bmatrix}
3\\
4
\end{bmatrix}
$
\quad
\text{b)}
\quad
	$
	\begin{bmatrix}
	1\\
	2\\
	3\\
	4
	\end{bmatrix},
	\begin{bmatrix}
	2\\
	3\\
	4\\
	5
	\end{bmatrix},
	\begin{bmatrix}
	3\\
	4\\
	5\\
	6
	\end{bmatrix}
	$
	\quad
\text{c)}
\quad
	$
	\begin{bmatrix}
	1\\
	2\\
	4
	\end{bmatrix},
	\begin{bmatrix}
	2\\
	3\\
	5
	\end{bmatrix},
	\begin{bmatrix}
	3\\
	4\\
	6
	\end{bmatrix}
	$
\end{oppgave}


\begin{oppgave}{E9}
Finn en vektor som ikke er en lineærkombinasjon av:
%
	$$
	\begin{bmatrix}
	1\\
	5\\
	-3
	\end{bmatrix},
	\begin{bmatrix}
	4\\
	18\\
	4
	\end{bmatrix}
	$$

\end{oppgave}


\begin{oppgave}{E10}
	Finn en tredje vektor i samme plan som disse to vektorene:
\[
	\begin{bmatrix}
	-3  \\
	-7 \\
	-3 \\
	\end{bmatrix}
	\quad\text{og}\quad
	\begin{bmatrix}
	8  \\
	-8 \\
	-4 \\
	\end{bmatrix}
\]
\end{oppgave}



%
%
%
%
%\begin{oppgave}
%	La $m<n$. Kan $m$ vektorer spenne ut $\mathbb{R}^n$? 
%\end{oppgave}
%
%

\begin{oppgave}{C11}
La $A$ og~$B$ være matriser, og $\V{v}$ en vektor:
\[
A =
\begin{bmatrix}
 0 & 1 &  5 \\
 2 & 3 & -1 \\
-8 & 0 & 2
\end{bmatrix}
\quad
B =
\begin{bmatrix}
1 & 2 & 5 \\
0 & 0 & 3
\end{bmatrix}
\quad
\V{v} = \vvv{7}{2}{-4}
\]
Regn ut, eller forklar hvorfor uttrykkene ikke gir mening:

\noindent
\begin{minipage}{0.14\textwidth}

a) $AB$


b) $BA$


c) $A^2$

\end{minipage}
\begin{minipage}{0.19\textwidth}

d) $B^2$


e) $A+B$


f) $(A + I_3) \V{v}$

\end{minipage}
\begin{minipage}{0.1\textwidth}

g) $BA\V{v}$


h) $B\tr$


i) $\V{v}\tr \V{v}$

\end{minipage}
\end{oppgave}



\begin{oppgave}{C12}
Løs likningen
$A\V{x}=\V{b}$  
der
\[
A=\begin{bmatrix}
	1 & 2 & 3\\
	2 & 3 & 4\\
	3 & 4 & 5\\
	4 & 5 & 6
	\end{bmatrix}
\quad
\V{b}_2=
	\begin{bmatrix}
	1  \\
	1 \\
	1 \\
	1
	\end{bmatrix}
\]
\end{oppgave}


\begin{oppgave}{C13}
Finn en kvadratisk matrise~$A$ slik at:
$A \vv{1}{0} = \vv{3}{5}$
og
$A \vv{1}{1} = \vv{-1}{0}$
\end{oppgave}


\begin{oppgave}{C14}
Bestem om matrisene er inverterbare, og finn om mulig den inverse matrisen.


a)
$\begin{bmatrix}
1 & 7 \\
-1 & 1 \\
\end{bmatrix}$



b)
$\begin{bmatrix}
1 & 0 & 0\\
1 & 1 & 0\\
1 & 1 & 1
\end{bmatrix}$


\end{oppgave}








\end{document}
