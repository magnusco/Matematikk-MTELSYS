\documentclass[a4paper,norsk,11pt]{interaktiv}

% Importerte pakker
\usepackage{float}
\restylefloat{figure}
\usepackage{ifluatex}
\usepackage{subfigure}
\usepackage{tikz}
\usetikzlibrary{arrows}
\usepackage[parfill]{parskip}    		% Activate to begin paragraphs with an empty line rather than an indent
\usepackage{graphicx}
\usepackage{tcolorbox}
\newcommand{\V}[1]{\mathbf{#1}}

\ifluatex
  \usepackage{fontspec}
  \setmainfont{Calibri}
  \usepackage{unicode-math}
  \setmathfont{Cambria Math}
\else
  \usepackage[utf8]{inputenc}
\fi

\usepackage[european,americanvoltages]{circuitikz}

\usepackage{url}

% Underfigurer
\renewcommand{\thesubfigure}{(\arabic{subfigure})}

% Overskrift
\emnekode{TMA4101}
\emnenavn{Matematikk 1 for MTELSYS}
\title{Likninger}

% Nye kommandoer
\newcommand{\dee}{\mathop{}\!{d}}

\begin{document}
\pagenumbering{gobble}

\maketitle


%\begin{tcolorbox}
%\end{tcolorbox}

Jacobis iterasjon osv er eksempler på numeriske metoder som produserer en følge.
Hvis alt er laget med omhu, 
konvergerer følgen til det vi vil ha. 


\begin{oppgave}{1}
Google ordet 'konvergens'.
\end{oppgave}




\begin{oppgave}{2}
Fikspunktiterasjon.
Finn ut hva som skjer med likningen $x=\cos{x}$ og iterasjonen $x_{n+1}=\cos{x_n}$.
\end{oppgave}


\begin{oppgave}{2}
$x\ln x=1$
\end{oppgave}



\begin{oppgave}{3}
Hvordan kan vi måle om vi er 'i mål'? Hint hint cauchyfølger.
\end{oppgave}



\end{document}
