\documentclass[a4paper,norsk,11pt]{interaktiv}

% Importerte pakker
\usepackage{float}
\restylefloat{figure}
\usepackage{ifluatex}
\usepackage{subfigure}
\usepackage{tikz}
\usetikzlibrary{arrows}
\usepackage[parfill]{parskip}    		% Activate to begin paragraphs with an empty line rather than an indent
\usepackage{graphicx}
\usepackage{tcolorbox}
\newcommand{\V}[1]{\mathbf{#1}}

\ifluatex
  \usepackage{fontspec}
  \setmainfont{Calibri}
  \usepackage{unicode-math}
  \setmathfont{Cambria Math}
\else
  \usepackage[utf8]{inputenc}
\fi

\usepackage[european,americanvoltages]{circuitikz}

\usepackage{url}

% Underfigurer
\renewcommand{\thesubfigure}{(\arabic{subfigure})}

% Overskrift
\emnekode{TMA4101}
\emnenavn{Matematikk 1 for MTELSYS}
\title{Areal}

% Nye kommandoer
\newcommand{\dee}{\mathop{}\!{d}}

\begin{document}
\pagenumbering{gobble}

\maketitle

Etter fullendt R2 sitter man gjerne igjen med følgende ide:

\begin{tcolorbox}
INTEGRASJON ER DET MOTSATTE AV DERIVASJON
\end{tcolorbox}

som de fleste i sitt stille sinn oversetter til:

\begin{tcolorbox}
INTEGRASJON = ANTIDERIVASJON
\end{tcolorbox}

Den første av disse er i bunn og grunn riktig, 
men ikke den andre.
Det som er riktig å si,
er at man i mange tilfeller kan antiderivere for å finne integralet.


\begin{oppgave}{1}
Finn arealet avgrenset av $y=x^3$, $y=0$ og $x=1$.
\end{oppgave}


I mange situasjoner er det enten umulig eller urealistisk å antiderivere for å finne integralet. 
Funksjonen $e^{-x^2}$ har ingen antiderivert som lar seg skrive ned på en pen måte, 
og andre funksjoner har så kompliserte antideriverte at det ikke er vits i å prøve en gang:
\begin{figure}[htbp]
  \begin{center}
%      \vspace*{-2.4cm}
	\includegraphics[scale=.80]{integral}
   % \vspace*{-2.4cm}
	\end{center}
\end{figure}

Men det viktigste å skjønne når vi skal forstå integrasjon, 
er hva en riemannsum er. 
Disse brukes egentlig til å definere integralet, 
men kan også brukes til å finne en tilnærming til arealet under grafen.

\begin{oppgave}{2}
Finn 
\[
\int_0^1 e^{-x^2}\; dx
\]
\end{oppgave}

Viggo Brun skal ha sagt:
\begin{tcolorbox}
Derivasjon er et håndverk, integrasjon er en kunst!
\end{tcolorbox}

Det han mente å si, 
var nok at derivasjon er ganske lett (det er bare å bruke noen regneregler),
mens integrasjon kan være ganske vanskelig. 
Et tilforlatelig funksjonsuttrykk kan være enten lett, umulig, vanskelig, eller håpløst å antiderivere,
og det kan være vanskelig å avgjøre hvilken av disse som er tilfelle. 
Man kan bruke et liv på å trene på antiderivasjon.

\begin{oppgave}{3}
  Regn ut integralene
  \begin{equation*}
    (i) \quad \int_0^{\frac{\pi}{2}} \sqrt{1 + \cos x} \dee x \qquad
    \qquad (ii) \quad \int\frac{\sin\left( x \right)}{1+\cos\left( x
      \right)} \dee x.
  \end{equation*}
  (Hint: $\cos(2\theta) = 2\cos^2 \theta -1 = 1 - 2 \sin^2 \theta$.)
\end{oppgave}





\end{document}
