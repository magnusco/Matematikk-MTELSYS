\documentclass[a4paper,norsk,11pt]{interaktiv}
\usepackage{tcolorbox}
% Importerte pakker
\usepackage{float}
\restylefloat{figure}
\usepackage{ifluatex}
\usepackage{subfigure}
\usepackage{tikz}
\usetikzlibrary{arrows}
\usepackage[parfill]{parskip}    		% Activate to begin paragraphs with an empty line rather than an indent
\usepackage{graphicx}
\usepackage[standard-baselineskips]{cmbright}

\newcommand{\V}[1]{\mathbf{#1}}
\usepackage{unicode-math}

\ifluatex
  \usepackage{fontspec}
  \setmainfont{Calibri}
  \usepackage{unicode-math}
  \setmathfont{Cambria Math}
\else
  \usepackage[utf8]{inputenc}
\fi

\usepackage[european,americanvoltages]{circuitikz}

\usepackage{url}

\newenvironment{amatrix}[1]{% "augmented matrix"
  \left[\begin{array}{*{#1}{c}|c}
}{%
  \end{array}\right]
}
% Underfigurer
\renewcommand{\thesubfigure}{(\arabic{subfigure})}

% Overskrift
\emnekode{TMA4101}
\emnenavn{Matematikk 1 for MTELSYS}




% Nye kommandoer
\newcommand{\dee}{\mathop{}\!{d}}


\title{Lineære likningssystemer III}



\begin{document}
\pagenumbering{gobble}

\maketitle

I denne øvingen skal vi se på en systematisk teknikk for å sette opp et lineært likningssystem for en resistiv krets,
altså en krets med kun motstander og spennings- eller strømkilder.

\section*{Standardoppgaver}

Maskestrømsmetoden består i å tenke at de ukjente er sirkulære strømmer som går i bane i hver sløyfe. 
Det er like mange ukjente som sløyfer i kretsen, 
og man får korrekt antall likninger ved å summere spenningsfallet rundt hver sløyfe
og bruke Kirchhoffs spenningslov.

\begin{oppgave}{S1}
Sett opp et likningssystem for kretsen.
\begin{center}
	\begin{circuitikz}
		\draw  (0,3) to [V=$1V$] (0,0);
		\draw  (0,3) to   [ R=$1\Omega$] (3,3) to [R=$1\Omega$] (3,0) to [R=$1\Omega$] (0,0); 
		\draw  (3,3) to   [ R=$1\Omega$] (6,3) to (6,0) to [R=$1\Omega$] (3,0); 
		\draw  (0,3) to (0,6) to   [ R=$1\Omega$] (3,6) to [R=$1\Omega$] (3,3); 
		\draw  (3,6) to   [ R=$1\Omega$] (6,6) to (6,3); 
		\end{circuitikz}
\end{center}
\end{oppgave}

\begin{oppgave}{S2}
Sett opp et likningssystem for kretsen.
\begin{center}
	\begin{circuitikz}
		\draw  (0,3) to [V=$V$] (0,0);
		\draw  (0,3) to   [ R=$R$] (3,3) to [R=$R$] (3,0) to [R=$R$] (0,0); 
		\draw  (3,3) to   [ R=$R$] (6,3) to (6,0) to [R=$R$] (3,0); 
		\draw  (0,3) to (0,6) to   [ R=$R$] (3,6) to [R=$R$] (3,3); 
		\draw  (3,6) to   [ R=$R$] (6,6) to (6,3); 
		\end{circuitikz}
\end{center}
\end{oppgave}

\begin{oppgave}{S3}
Sett opp et likningssystem for kretsen.
\begin{center}
	\begin{circuitikz}
		\draw  (0,3) to [V=$V$] (0,0);
		\draw  (0,3) to   [ R=$R_1$] (3,3) to [R=$R_2$] (3,0) to [R=$R_3$] (0,0); 
		\draw  (3,3) to   [ R=$R_4$] (6,3) to (6,0) to [R=$R_5$] (3,0); 
		\draw  (0,3) to (0,6) to   [ R=$R_6$] (3,6) to [R=$R_7$] (3,3); 
		\draw  (3,6) to   [ R=$R_8$] (6,6) to (6,3); 
		\end{circuitikz}
\end{center}
\end{oppgave}

\begin{oppgave}{S4}
Sett opp et likningssystem for kretsen.
\begin{center}
	\begin{circuitikz}
		\draw  (0,3) to [V=$V_1$] (0,0);
		\draw  (0,3) to [R=$R_1$] (3,3) to [R=$R_2$] (3,0) to [R=$R_3$] (0,0); 
		\draw  (3,3) to [R=$R_4$] (6,3);
		\draw (3,0) to [R=$R_5$] (6,0); 
		\draw  (6,0) to [V=$V_2$] (6,3);
		\draw  (0,6) to [R=$R_6$] (3,6) to [R=$R_7$] (3,3); 
		\draw  (0,6) to [V=$V_3$] (0,3);
		\draw  (3,6) to [R=$R_8$] (6,6); 
		\draw (6,3) to [V=$V_4$] (6,6);
		\end{circuitikz}
\end{center}
\end{oppgave}


\section*{Viderekomne oppgaver}

Fordelen med maskestrømmer, 
er at det blir få likninger.
Ulempen er at kretsen må være planar.
Noen synes også maskestrømsmetoden er litt shady, 
for du kan potensielt ha ukjente som ikke kan måles med et multimeter.

Nodespenningsmetoden er en annen systematisk metode for å sette opp lineære likningssystemer fra et kretsdiagram. 
Det bli litt flere likninger enn for maskestrømsmetoden, 
men den kan brukes selv om kretsen ikke er planar,
og du kan verifisere at du har regnet riktig ved å måle de ukjente med et multimeter, 
noe som er behagelig for de som ikke tror på noe før de har målt det fysisk.

Nodespenningsmetoden består i å sette opp spenningen i hver node (relativt til jordsymbolet) som ukjent i hver node, 
og bruke Kirchhoffs strømlov på hver node.  
Det er lett å se at man også her får like mange ukjente som likninger.


\begin{oppgave}{V1-4}
Sett opp et likningssystem for kretsen. Fortsett så med kretsene fra S2-S4 over. 
\begin{center}
	\begin{circuitikz}
		\draw  (0,3) to [V=$1V$] (0,0);
		\draw  (0,3) to   [ R=$1\Omega$] (3,3) to [R=$1\Omega$] (3,0) to [R=$1\Omega$] (0,0); 
		\draw  (3,3) to   [ R=$1\Omega$] (6,3) to (6,0) to [R=$1\Omega$] (3,0); 
		\draw  (0,3) to (0,6) to   [ R=$1\Omega$] (3,6) to [R=$1\Omega$] (3,3); 
		\draw  (3,6) to   [ R=$1\Omega$] (6,6) to (6,3); 
		\draw (.4,0) node[ground]{};
		\end{circuitikz}
\end{center}
\end{oppgave}

\begin{oppgave}{V5}
Nodespenningsmetoden gir alltid flere likninger og ukjente enn maskestrømsmetoden.
\end{oppgave}




\end{document}
