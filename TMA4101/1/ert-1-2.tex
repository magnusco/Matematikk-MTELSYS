\documentclass[a4paper,norsk,11pt]{interaktiv}
\usepackage{tcolorbox}
% Importerte pakker
\usepackage{float}
\restylefloat{figure}
\usepackage{ifluatex}
\usepackage{subfigure}
\usepackage{tikz}
\usetikzlibrary{arrows}
\usepackage[parfill]{parskip}    		% Activate to begin paragraphs with an empty line rather than an indent
\usepackage{graphicx}
\usepackage[standard-baselineskips]{cmbright}

\newcommand{\V}[1]{\mathbf{#1}}
\usepackage{unicode-math}

\ifluatex
  \usepackage{fontspec}
  \setmainfont{Calibri}
  \usepackage{unicode-math}
  \setmathfont{Cambria Math}
\else
  \usepackage[utf8]{inputenc}
\fi

\usepackage[european,americanvoltages]{circuitikz}

\usepackage{url}

\newenvironment{amatrix}[1]{% "augmented matrix"
  \left[\begin{array}{*{#1}{c}|c}
}{%
  \end{array}\right]
}
% Underfigurer
\renewcommand{\thesubfigure}{(\arabic{subfigure})}

% Overskrift
\emnekode{TMA4101}
\emnenavn{Matematikk 1 for MTELSYS}




% Nye kommandoer
\newcommand{\dee}{\mathop{}\!{d}}


\title{Lineære likningssystemer II}


\begin{document}
\pagenumbering{gobble}

\maketitle

Nå skal vi se på noen nye likningssystemer.


\begin{oppgave}{1}
Amund har høns og kyr på gården sin. Vrang av vane vil han ikke si hvor mange dyr han har, men opplyser heller at de har 382 bein og 141 hoder. Hvor mange høns og hvor mange kyr har Amund?
\end{oppgave}

\begin{oppgave}{2}
Løs systemet
\[
\setlength\arraycolsep{1pt}
\begin{array}{rcrcrc@{\;}l}
  2x_1   & + &   3x_2   & + &      4x_3  &  =   &  4 \\
   3x_1   & + &  4x_2   & + &       5x_3   &  =   &  5 \\
  4x_1   &  + &    5 x_2   &  + &   7x_3   &  =   &  3 
\end{array}
\]
\end{oppgave}


\begin{oppgave}{3}
Løs systemet
\[
\setlength\arraycolsep{1pt}
\begin{array}{rcrcrc@{\;}l}
  2x_1   & + &   3x_2   & + &      4x_3  &  =   &  4 \\
   3x_1   & + &  4x_2   & + &       5x_3   &  =   &  5 \\
  4x_1   &  + &    5 x_2   &  + &   6x_3   &  =   &  3 
\end{array}
\]
\end{oppgave}


\begin{oppgave}{4}
Løs systemet
\[
\setlength\arraycolsep{1pt}
\begin{array}{rcrcrc@{\;}l}
  2x_1   & + &   3x_2   & + &      4x_3  &  =   &  4 \\
   3x_1   & + &  4x_2   & + &       5x_3   &  =   &  5 \\
  4x_1   &  + &    5 x_2   &  + &   6x_3   &  =   &  6 
\end{array}
\]
\end{oppgave}


\begin{oppgave}{5}
Løs systemet
\[
\setlength\arraycolsep{1pt}
\begin{array}{rcrcrc@{\;}l}
  2x_1   & + &   3x_2   & + &      4x_3  &  =   &  0 \\
   3x_1   & + &  4x_2   & + &       5x_3   &  =   &  0 \\
  4x_1   &  + &    5 x_2   &  + &   6x_3   &  =   &  0 
\end{array}
\]
\end{oppgave}

\begin{oppgave}{6}
Finn et andregradspolynom som går gjennom punktene $(1,2)$, $(2,3)$ og $(3,5)$.
\end{oppgave}

\end{document}
