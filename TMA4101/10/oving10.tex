\documentclass[a4paper,norsk,11pt]{interaktiv}

% Importerte pakker
\usepackage{float}
\restylefloat{figure}
\usepackage{ifluatex}
\usepackage{subfigure}
\usepackage{tikz}
\usetikzlibrary{arrows}
\usepackage[parfill]{parskip}    		% Activate to begin paragraphs with an empty line rather than an indent
\usepackage{graphicx}
\usepackage[american]{circuitikz}


\ifluatex
  \usepackage{fontspec}
  \setmainfont{Calibri}
  \usepackage{unicode-math}
  \setmathfont{Cambria Math}
\else
  \usepackage[utf8]{inputenc}
\fi

% Underfigurer
\renewcommand{\thesubfigure}{(\arabic{subfigure})}

% Overskrift
\emnekode{TMA4101}
\emnenavn{Matematikk 1 for MTELSYS}
\title{Øving 10 - Differensiallikninger}
%\author{Fourieranalyse}


% Nye kommandoer
\newcommand{\dee}{\mathop{}\!{d}}

\begin{document}
\pagenumbering{gobble}

\maketitle


\section*{Obligatoriske oppgaver}


\begin{oppgave}{1}
Delbrøksoppspaltning er en teknikk for å spalte store stygge brøker.
Denne undervises gjerne i forbindelse med integrasjon eller laplacetransform.
Å lære seg å spalte alle verdens brøker er en del jobb,
og ikke kjempenyttig, 
men står i Adams på s. 344. 
I denne oppgaven skal vi spalte
\[
\frac{1}{(s+1)(s^2+1)}=\frac{A}{s+1}+\frac{Bs+C}{s^2+1}.
\]
Finn $A$, $B$ og $C$. 

(Hint: gang opp med $(s+1)(s^2+1)$ og sammenlikne begge sider av den resulterende likningen.
Det er også mulig å spalte slik:
\[
\frac{1}{(s+1)(s^2+1)}=\frac{1}{(s+1)(s+i)(s-i)}=\frac{A}{s+1}+\frac{B}{s+i}+\frac{C}{s-i}
\]
men dette blir mer jobb.)
\end{oppgave}


\begin{oppgave}{2}
Dersom du summerer spenningsfallet over kretsen under, 
vil du få en variant av likningen 
\[
y'(t)+y(t)=f(t).
\]
Finn den analytiske løsningen når $f(t)=\sin t$.
\begin{center}
	\begin{circuitikz}
		\draw  (0,1) to [V=$f(t)$] (0,3) to [R ] (4,3) to [C] (4,1) to  (0,1);
	\end{circuitikz}
	
%	\begin{circuitikz}
%		\draw (1.1,1) to [short, o-] (0,1) to [R = $R$] (0,4) to[C=$C$] (3,4) to [L = $L$] (3,1) to  [short, -o] (1.9,1);
%		\draw (1.5, 1) node [below] {$v(t)$};
%	\end{circuitikz}
\end{center}
\end{oppgave}

\section*{Anbefalte oppgaver}

%\begin{oppgave}{E2}
%Dersom du summerer spenningsfallet over kretsen under, 
%vil du få en variant av likningen 
%\[
%y'(t)+y(t)+\exp y(t)=f(t)
%\]
%\begin{center}
%	\begin{circuitikz}
%		\draw  (0,1) to [V=$f(t)$] (0,3) to [R ] (4,3) to [C] (4,1) [D] to (0,1);
%	\end{circuitikz}
%	
%%	\begin{circuitikz}
%%		\draw (1.1,1) to [short, o-] (0,1) to [R = $R$] (0,4) to[C=$C$] (3,4) to [L = $L$] (3,1) to  [short, -o] (1.9,1);
%%		\draw (1.5, 1) node [below] {$v(t)$};
%%	\end{circuitikz}
%\end{center}
%Lag et pythonscript som løser denne likningen numerisk for $f(t)=1$ og $y(0)=1$ med 
%\begin{itemize}
%\item[a)] Eulers eksplisitte metode.
%\item[b)] Eulers implisitte metode.
%\end{itemize}
%\end{oppgave}







\begin{oppgave}{1}
  Løs initialverdiproblemet
  \begin{equation*}
    2y'=xy, \qquad y(0)=-2.
  \end{equation*}
\end{oppgave}

\begin{oppgave}{2}
    Løs initialverdiproblemet
    \begin{equation*}
      y' = 1-y^2, \qquad y(0)=0.5
    \end{equation*}
\end{oppgave}

\begin{oppgave}{3}
Newtons avkjølingslov sier at temperaturen $T$ i en kokt elgtunge som settes inn i kjøleskapet (eller ut av kjøleskapet, eller inn i stekeovnen osv), 
følger initialverdiproblemet
\[
T'(t)+\alpha \left(T(t)-T_k\right)=0 \quad T(0)=T_0,
\]
der $t$ er tiden, $T_0$ er temperaturen til elgtungen i det den settes inn  i kjøleskapet, 
$T_k$ er temperaturen i kjøleskapet,
og $\alpha$ er en konstant som avhenger av varmeflyten mellom elgtungen og omgivelsene. 

  En melkekartong der temperaturen i melken var $6^\circ$C, ble
  stående på kjøkkenbenken i $2$ timer. Da var temperaturen steget til
  $13^\circ$C. Lufttemperaturen i kjøkkenet var $20^\circ$C. Vi regner
  med at Newtons avkjølingslov gjelder, det vil si at
  temperaturendringen per tidsenhet i melken er proporsjonal med
  differansen mellom lufttemperaturen og temperaturen i melken.

  % a)
  \bPunkt
    Sett opp en differensialligning for temperaturen $T$ i melken som
    funksjon av tiden $t$, og vis at den har løsning av formen
    \begin{equation*}
      T(t) = A+ Be^{- \alpha t}
    \end{equation*}
    der $A$ er lufttemperaturen. Finn konstantene $B$ og $\alpha$.
  \ePunkt

  % b)
  \bPunkt
    Da temperaturen i melken var $15^\circ$C, ble kartongen satt
    inn i kjøleskapet. Etter $1$ time var temperaturen i melken sunket
    til $12^\circ$C. Hva var temperaturen i kjøleskapet?
  \ePunkt
\end{oppgave}
%
%

\begin{oppgave}{4}
  Finn en ligning for en kurve som passerer gjennom $(2,3)$ og har
  stigning $\displaystyle\frac{2x}{1+y^2}$.
  \\[6pt]
\end{oppgave}



%\begin{oppgave}{E1}
%En bestemt virusepidemi i et ikke navngitt samfunn er det slik at hver smittede person smitter to nye hver dag. 
%Dersom viruset introduseres av en person som kommer hjem fra skiferie, 
%hvor mange dager tar det før tusen mennesker er smittet?
%Hvor mange dager tar det før en million er smittet?
%\end{oppgave}
%
%
%\begin{oppgave}{A1}
%Utled løsningsuttrykket 
%\[
%y(t)=
%\]
%for 
%\[
%y'(t)+y(t)=f(t)
%\]
%for en generell $f$.
%\end{oppgave}




\end{document}
