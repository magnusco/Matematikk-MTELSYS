\documentclass[a4paper,norsk,11pt]{interaktiv}

% Importerte pakker
\usepackage{float}
\restylefloat{figure}
\usepackage{ifluatex}
\usepackage{subfigure}
\usepackage{tikz}
\usetikzlibrary{arrows}
\usepackage[parfill]{parskip}    		% Activate to begin paragraphs with an empty line rather than an indent
\usepackage{graphicx}



\ifluatex
  \usepackage{fontspec}
  \setmainfont{Calibri}
  \usepackage{unicode-math}
  \setmathfont{Cambria Math}
\else
  \usepackage[utf8]{inputenc}
\fi

% Underfigurer
\renewcommand{\thesubfigure}{(\arabic{subfigure})}

% Overskrift
\emnekode{TMA4101}
\emnenavn{Matematikk 1 for MTELSYS}
\title{Øving 9 - Integralet}
%\author{Fourieranalyse}


% Nye kommandoer
\newcommand{\dee}{\mathop{}\!{d}}

\begin{document}
\pagenumbering{gobble}

\maketitle


\section*{Obligatoriske oppgaver}


\begin{oppgave}{1}
  En beholder med høyde $4$ lages ved å rotere kurven
  $y = x^2$, $0 \leq x \leq 2$, om aksen $x = -1$ og sette en
  plan bunn i.  Finn volumet $V$ av beholderen.
\end{oppgave}

\begin{oppgave}{2}
  La funksjonen $F$ være definert for $x \geq 1$ ved
  \begin{equation*}
    F(x) = \int_1^x \sqrt{t^3 - 1} \dee t,
  \end{equation*}
  og la $K$ være kurven $y = F(x)$ for $1 \leq x \leq 2$.  Finn
  buelengden av $K$.
\end{oppgave}




\section*{Anbefalte oppgaver}

\begin{oppgave}{1}
  % a)
  \bPunkt
    Et vannkar dannes ved å rotere kurven
    \begin{equation*}
      y = \frac{x^3}{4}, \qquad x \geq 0
    \end{equation*}
    om $y$-aksen. Finn volumet av karet opp til høyde $h$.
  \ePunkt

  % b)
  \bPunkt
    Karet fylles med vann. Hvor fort stiger vannhøyden i karet
    idet høyden er $2$ dm og vannet strømmer inn med $10$ liter per
    sekund? (Vi antar at $x$ og $y$ er målt i dm.)
  \ePunkt
\end{oppgave}

\begin{oppgave}{2}
  Et $45^\circ$ hakk karves inn til midten av en sylindrisk kubbe som
  er $40$cm tykk, slik som i Figur $7.20$ på s.\ $406$ i læreboka. Den
  ene overflaten til hakket er vinkelrett på aksen som går gjennom
  kubben. Hvor mye tre (i volum) ble fjernet ved å lage dette hakket?
\end{oppgave}

\begin{oppgave}{3}
  La $f(x)$ være en ikkenegativ funksjon som er deriverbar med
  kontinuerlig derivert for $x \geq 1$. Buelengden til kurven
  $y = f(x)$ fra $x = 1$ til $x = u$ er gitt ved en funksjon
  $H(u)$. Bestem funksjonen $f$ dersom
  \begin{equation*}
    H(u) = \frac{u^3}{3} + u -\frac{4}{3} \quad \text{og} \quad f(1) =
    0.
  \end{equation*}
\end{oppgave}

\begin{oppgave}{4}
  Lisa skal bake smultringer på dugnad for UKA-$19$. Smultringene skal
  være $2$cm tykke, ha senterhull på $2$cm i diameter og dekkes med
  sjokoladeglasur. Det trengs $5$dl glasur for å dekke $1$
  kvadratmeter bakverk. Hvor mye sjokoladeglasur trenger hun om hun
  skal lage 100 smultringer?
\end{oppgave}

\begin{oppgave}{5}
  Et av svømmebassengene på Pirbadet er $20$ meter langt, $8$ meter
  bredt og har en skrå bunn som er slik at dybden ved den ene
  kortsiden er $1$ meter, og $3$ meter ved den andre. Finn den totale
  kraften som virker på bassengbunnen, som følge av trykket i væsken,
  når bassenget er fylt opp med vann.
\end{oppgave}


\begin{oppgave}{6}
  La $S$ være området i $xy$-planet som er avgrensa av kurvene $y=x^2$
  og $y=\sqrt{x}$ mellom $x=0$ og $x=1$. Finn volumet av
  omdreiningslegemet som oppstår ved å dreie $S$ om $x$-aksen. Bruk
  både sylinderskallmetoden og skivemetoden.\\[-6pt]
\end{oppgave}

\begin{oppgave}{7}
  La $R$ være området avgrenset av $y=x$ og $y=x^2$ mellom $x=0$ og
  $x=1$.

  \setcounter{Punkt}{0}

  % a)
  \bPunkt
    Finn volumet av omdreiningslegemet som oppstår når $R$ blir rotert
    om $x$-aksen.
  \ePunkt

  % b)
  \bPunkt
    Finn volumet av omdreiningslegemet som oppstår når $R$ blir rotert
    om $y$-aksen.
  \ePunkt
\end{oppgave}

\begin{oppgave}{8}
  Et legeme er 6 meter høyt. Det horisontale tversnittet i en høyde
  $z$ meter over grunnflata er et rektangel med lengde $2 + z$ meter og
  bredde $8 - z$ meter. Finn volumet av legemet.\\[-6pt]
\end{oppgave}

\begin{oppgave}{9}
  Finn buelengden til kurven $$ y^2= (x-1)^3$$ fra $(1,0)$ til
  $(2,1)$.\\[-6pt]
\end{oppgave}

\begin{oppgave}{10}
  Finn arealet av overflata som oppstår ved å rotere kurven
  $$y=x^2, \quad 0\leq x \leq 2$$ om $y$-aksen.\\[-6pt]
\end{oppgave}

\begin{oppgave}{11}
  La $A$ være området i $xy$-planet som er avgrensa av kurvene
  \begin{align*}
    &y_1=4-x^2 \\
    &y_2=2-x.
  \end{align*}
  Bestem volumet av omdreiningslegemet som oppstår ved å dreie $A$ om
  $x$-aksen. \\[3mm] (H16) \\[-6pt]
\end{oppgave}

\begin{oppgave}{12}
  La $a$ og $h$ være positive størrelser, og la $A$ være området i
  $1$. kvadrant avgrenset av parabelen $y=ax^2$, $y$-aksen og den
  horisontale linjen $y=h$. Området $A$ roteres om $y$-aksen.

  \setcounter{Punkt}{0}

  % a)
  \bPunkt
    Området $A$ roteres om $y$-aksen. Finn volumet til rotasjonslegemet.
  \ePunkt

  % b)
  \bPunkt
    Anta at rotasjonslegemet er fylt med vann som deretter tappes ut. I
    et gitt øyeblikk er vannhøyden 1 m, og vannet strømmer ut med en
    hastighet av $2$ dm$^3$ pr. sekund. Hvis $a= \pi $(dm$^{-1}$), hvor
    raskt avtar vannhøyden i dette øyeblikket?
  \ePunkt
  (Sommer 2013)\\[-6pt]
\end{oppgave}

%
%\begin{oppgave}{1}
%  Bruk delvis integrasjon til å beregne integralet 
%  \begin{equation*}
%    \int e^{2x} \sin(3x) \dee x.
%  \end{equation*}
%\end{oppgave}
%
%\begin{oppgave}{2}
%  Regn ut integralet 
%  \begin{equation*}
%    \int \frac{x^2 \dee x}{x^2 +3x +2}.
%  \end{equation*}
%\end{oppgave}
%
%\begin{oppgave}{3}
%  Avgjør om integralet 
%  \begin{equation*}
%    \int_{-1}^\infty \frac{x +3}{x^2 +4x +4} \dee x
%  \end{equation*}
%  konvergerer eller divergerer.
%\end{oppgave}
%
%\begin{oppgave}{4}
%  % a)
%  \bPunkt
%    Vis at 
%    \begin{equation*}
%      \lim_{x \to 0+} x^c \ln x = 0
%    \end{equation*}
%    når $c > 0$.
%  \ePunkt
%
%  % b)
%  \bPunkt
%    Regn ut det uegentlige integralet 
%    \begin{equation*}
%      \int_0^1 x^a \ln x \dee x
%    \end{equation*}
%    når $a > -1$. Hva kan du si om integralet når $a = -1$?
%  \ePunkt
%\end{oppgave}
%
%\begin{oppgave}{5}
%  Regn ut de ubestemte integralene 
%  \begin{equation*}
%    (i) \int \frac{1 + \sqrt{x}}{1+ \sqrt[3]{x}} \dee x \hspace{1.5cm}
%    (ii)  \int \frac{\dee\theta}{3+ 2\cos \theta}. 
%  \end{equation*}
%\end{oppgave}
%
%Regn ut integralene. (NB: noen av dem er ganske vanskelige.)
%
%  \begin{oppgave}{1}
%    $\displaystyle\quad \int (x^2-2x)e^{kx}\dee x.$
%  \end{oppgave}
%  
%  
%  \begin{oppgave}{2}
%    $\displaystyle\quad \int (\arcsin x)^2\dee x.$
%  \end{oppgave}
%  
%  \begin{oppgave}{3}
%    $\displaystyle\quad \int \frac{x \dee x }{3x^2+8x-3}.$
%  \end{oppgave}
%  
%  
%  \begin{oppgave}{4}
%    $\displaystyle\quad \int \frac{\dee x}{x^4-a^4}.$
%  \end{oppgave}
%  
%  
%  \begin{oppgave}{5}
%    $\displaystyle\quad \int \frac{\dee x}{x\sqrt{9-x^2}}.$
%  \end{oppgave}
%  
%  
%  \begin{oppgave}{6}
%    $\displaystyle\quad \int_0^{\pi/2} \frac{ d\theta}{1+\cos \theta +\sin\theta}.$ 
%  \end{oppgave}
%  
%  \begin{oppgave}{7}
%    $\displaystyle\quad \int_e^{\infty} \frac{\dee x}{x\ln x}.$
%  \end{oppgave}
%  
%  
%  \begin{oppgave}{8}
%    $\displaystyle\quad \int_2^{\infty} \frac{\dee x}{\sqrt{x}\ln x}.$
%  \end{oppgave}




\end{document}
