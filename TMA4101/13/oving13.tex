\documentclass[a4paper,norsk,11pt]{interaktiv}

% Importerte pakker
\usepackage{float}
\restylefloat{figure}
\usepackage{ifluatex}
\usepackage{subfigure}
\usepackage{tikz}
\usetikzlibrary{arrows}
\usepackage[parfill]{parskip}    		% Activate to begin paragraphs with an empty line rather than an indent
\usepackage{graphicx}
\usepackage[american]{circuitikz}
\usepackage{url}

\ifluatex
  \usepackage{fontspec}
  \setmainfont{Calibri}
  \usepackage{unicode-math}
  \setmathfont{Cambria Math}
\else
  \usepackage[utf8]{inputenc}
\fi

% Underfigurer
\renewcommand{\thesubfigure}{(\arabic{subfigure})}

% Overskrift
\emnekode{TMA4101}
\emnenavn{Matematikk 1 for MTELSYS}
\title{Øving 13 - Anvendelser av taylors teorem og taylorrekker}
%\author{Fourieranalyse}


% Nye kommandoer
\newcommand{\dee}{\mathop{}\!{d}}

\begin{document}
\pagenumbering{gobble}

\maketitle


\section*{Obligatoriske oppgaver}


\begin{oppgave}{1}
Funksjonene
$
f(x)=x^2
$
og 
$
g(x)=\sin x
$
skjærer hverandre i et punkt i intervallet $[0.5,1]$. 
Dersom man ønsker å finne denne ved fikspunktiterasjon, 
har man valget mellom å skrive likningen om til 
\[
x=\sqrt{\sin x}
\]
eller 
\[
x=\arcsin\left(x^2\right)
\]
Den ene av disse gjør jobben, og den andre gjør det ikke. 
Avgjør hvilken av dem som funker.
\end{oppgave}

\begin{oppgave}{2}
Vi skal tilnærme den deriverte til funksjonen $\sin x$ i punktet $x=1$ med formelen 
\begin{equation*}
f'(x) \approx \frac{f(x+h)-f(x-h)}{2h}.
\end{equation*} 
Bruk formelen for $h=0.1$,  $h=0.01$,  $h=0.001$ osv. 
Hvor høy presisjon klarer du å oppnå med denne strategien, 
og hva er den minste $h$ som er vits i å bruke? 
Kan du forklare hva som skjer?
\end{oppgave}




\section*{Anbefalte oppgaver}

\begin{oppgave}{1}
Finn en approksimasjon til $f'(x)$ basert på punktene $x$, $x-h$ og $x-2h$. 
Approksimasjonen skal ha så høy orden som mulig.
\end{oppgave}

\begin{oppgave}{2}
Vis at sentraldifferansen 
\begin{equation*}
f'(x) \approx \frac{f(x+h)-f(x-h)}{2h}
\end{equation*} 
har orden $h^2$.
\end{oppgave}

\begin{oppgave}{3}
Hvilken derivert er 
\begin{equation*}
\frac{-f(x-2h)+16f(x-h)-30f(x)+16f(x+h)-f(x+2h)}{12h^2}
\end{equation*}
en tilnærming til, og hva er ordenen?
\end{oppgave}

\begin{oppgave}{4}
Vi skal tilnærme den deriverte til funksjonen $\sin x$ i punktet $x=1$ med formelen 
\begin{equation*}
f'(x) \approx \frac{f(x+h)-f(x-h)}{2h}.
\end{equation*} 
Men nå skal vi bruke Richardson-ekstrapolasjon. 
Hvor høy presisjon klarer du å oppnå?
\end{oppgave}


\begin{oppgave}{5}
Lag et script som løser likningen $x=\cos x$ til maskinpresisjon med den vanlige fikspunktiterasjonen $x_{n+1}=\cos(x_n)$. Hvor mange iterasjoner trengs med startverdi $x=1$?
\end{oppgave}

\begin{oppgave}{6}
Lag et script som løser likningen $x=\cos x$ til maksinpresisjon med Newtons metode. Hvor mange iterasjoner trengs med startverdi $x_0=1$?
\end{oppgave}


\begin{oppgave}{7}
Bruk Newtons metode til å finne $\sqrt[3]{7}$ til maskinpresisjon. 
\end{oppgave}



\begin{oppgave}{8}
Løs likningen $x \ln x=1$ til maskinpresisjon med den klassiske fikspunktiterasjonen. 
Merk at likningen kan skrives om til $x=g(x)$ på flere måter.
\end{oppgave}

\begin{oppgave}{9}
Løs likningen $x \ln x=1$ til maskinpresisjon med Newtons metode. 
\end{oppgave}

\begin{oppgave}{10}
Løs likningen $x^3-x^2+x+2$ til maksinpresisjon med Newtons metode og startverdi $x=-1$. Gi et a priori estimat for hvor mange iterasjoner som trengs.
\end{oppgave}

\end{document}
