\documentclass[a4paper,norsk,11pt]{interaktiv}

% Importerte pakker
\usepackage{float}
\restylefloat{figure}
\usepackage{ifluatex}
\usepackage{subfigure}
\usepackage{tikz}
\usetikzlibrary{arrows}
\usepackage[parfill]{parskip}    		% Activate to begin paragraphs with an empty line rather than an indent
\usepackage{graphicx}


\ifluatex
  \usepackage{fontspec}
  \setmainfont{Calibri}
  \usepackage{unicode-math}
  \setmathfont{Cambria Math}
\else
  \usepackage[utf8]{inputenc}
\fi

% Underfigurer
\renewcommand{\thesubfigure}{(\arabic{subfigure})}

% Overskrift
\emnekode{TMA4101}
\emnenavn{Matematikk 1 for MTELSYS}
\title{Øving 8 - Integralet}
%\author{Fourieranalyse}


% Nye kommandoer
\newcommand{\dee}{\mathop{}\!{d}}

\begin{document}
\pagenumbering{gobble}

\maketitle


\section*{Obligatoriske oppgaver}


\begin{oppgave}{E1}
Funksjonen $f(x)=\exp{-x^2}$ er riemannintegrerbar, 
men har ingen antiderivert som kan skrives ved et endelig antall elementære funksjoner.
Lag et pythonscript som beregner en tilnærming til 
 \[
\int_0^1\exp{(-x^2)}\;dx=0.7468241328...
 \]
 ved  
  \bPunkt
  riemannsummer.  
   \ePunkt
   
     \bPunkt
 trapesmetoden. 
   \ePunkt
   
 Hvor mange riktige desimaler klarer du å få til med hver metode?
\end{oppgave}

\setcounter{Punkt}{0}

\begin{oppgave}{E2}
 Regn ut for alle $r\in \mathbb R$
  
  % a)
  \bPunkt
    $\displaystyle\int_0^1 x^{-r} \dee x$
  \ePunkt

  % b)
  \bPunkt
    $\displaystyle\int_1^\infty x^{-r} \dee x$.
  \ePunkt
\end{oppgave}



%\begin{oppgave}{E2}
%Funksjonen $f(x)=\frac{\log \cos x}{\sin 2x}$ er riemannintegrerbar, 
%og har en antiderivert, men denne tar rundt en A4-side å skrive opp. 
%Lag et script som beregner en tilnærming til 
% \[
%\int_0^1\frac{\log \cos x}{\sin 2x}\;dx=0.7468241328...
% \]
% ved trapesmetoden.
%\end{oppgave}



\section*{Anbefalte oppgaver}

\begin{oppgave}{B1}
  Finn en funksjon $f$ slik at
  \begin{equation*}
    \frac{1}{n^2} \sum_{k=1}^n k e^{- \frac{k^2}{n^2}}
  \end{equation*}
  er en Riemannsum for $f$ på intervallet $[0,1]$. Bruke dette til å
  bestemme grenseverdien
  \begin{equation*}
    \lim_{n \to \infty} \frac{1}{n^2} \sum_{k=1}^n k e^{-
      \frac{k^2}{n^2}}.
  \end{equation*}
\end{oppgave}

\begin{oppgave}{D2}
  Finn gjennomsnittsverdien av funksjonen $f(x) = |x+1| (u(x)-u(-x))$,  
  på intervallet $[-2,2]$, der  $u$ er heavisidefunksjonen.
  
%  
%   \begin{equation*}
%u(x)-u(-x) = \begin{cases} 
%      1 & x \geq 0 \\
%      -1 & x < 0 \\
%            0 & x = 0 
%    \end{cases}
%  \end{equation*}
\end{oppgave} 

\begin{oppgave}{C3}
  La
  \begin{equation*}
    F(t) = \int_0^t \cos(x^2) \dee x,
  \end{equation*}
  og finn $\frac{\dee}{\dee x} F(\sqrt{x})$.
\end{oppgave}

\begin{oppgave}{A4}
  Regn ut integralene
  \begin{equation*}
    (i) \quad \int_0^{\frac{\pi}{2}} \sqrt{1 + \cos x} \dee x \qquad
    \qquad (ii) \quad \int\frac{\sin\left( x \right)}{1+\cos\left( x
      \right)} \dee x.
  \end{equation*}
  (Hint: $\cos(2\theta) = 2\cos^2 \theta -1 = 1 - 2 \sin^2 \theta$.)
\end{oppgave}

\begin{oppgave}{B5}
  Finn området avgrenset av den lukkede kurven
  $y^{2}=x^{4}\left( 2+x \right)$ til venstre for $y$-aksen.
\end{oppgave}


\begin{oppgave}{C6}
  Regn ut 
  \begin{equation*}
    \lim_{n\to \infty} \sum_{i=1}^n \frac{2n+3i}{n^2}.
  \end{equation*}
  \\[-6pt]\\[-6pt]
\end{oppgave}

\begin{oppgave}{A7}
  Oppgave 5.3.8 fra Adams.\\
  La $P_n$ være en uniform partisjon av intervallet
  $\left[ 0,2 \right]$ hvor inkrementene har lengde $2/n.$ Regn ut
  $L\left( f,P_n \right)$ og $U\left( f,P_n \right)$ for funksjonen
  $f\left( x \right)=1-x$ på intervallet $\left[ 0,2 \right].$
  Deretter vis at
  \begin{equation*}
    \lim_{n\to \infty}L\left( f,P_n
    \right)=\lim_{n\to \infty}U\left( f,P_n \right).
  \end{equation*}
  \\[-6pt]
\end{oppgave}


\begin{oppgave}{E8}
  Finn gjennomsnittsverdien til $f(x) = e^{3x}$ på intervallet
  $[-2,2]$.
  \\[-6pt]
\end{oppgave}

\begin{oppgave}{A9}
  Regn ut
  \begin{equation*}
    \int \frac{x^2}{2+x^6} \dee x.
  \end{equation*}
  \\[-6pt]
\end{oppgave}

\begin{oppgave}{C10}
  Finn arealet begrenset av $y = x/(x^2 + 16)$, $y = 0$, $x = 0$ og $x
  = 2$.
  \\[-6pt]
\end{oppgave}

\begin{oppgave}{C11}
  Funksjonene $ y =\sin^2(x)$ og $y=1$ avgrenser et uendelig antall
  noe avrundede pizzastykker. Finn arealet av ett av disse.
  \\[-6pt]
\end{oppgave}

\begin{oppgave}{D12}
  Regn ut
  \begin{equation*}
    \int e^x \sqrt{1+e^x}\dee x.
  \end{equation*}
  (H14)
  \\[-6pt]
\end{oppgave}

\begin{oppgave}{B13}
  Regn ut
  \begin{equation*}
    \int_0^1 e^{\arcsin x} \dee x.
  \end{equation*}
\end{oppgave}




\end{document}
