\documentclass[a4paper,norsk,11pt]{interaktiv}

% Importerte pakker
\usepackage{float}
\restylefloat{figure}
\usepackage{ifluatex}
\usepackage{subfigure}
\usepackage{tikz}
\usetikzlibrary{arrows}
\usepackage[parfill]{parskip}    		% Activate to begin paragraphs with an empty line rather than an indent
\usepackage{graphicx}
\usepackage{tcolorbox}
\newcommand{\V}[1]{\mathbf{#1}}

\ifluatex
  \usepackage{fontspec}
  \setmainfont{Calibri}
  \usepackage{unicode-math}
  \setmathfont{Cambria Math}
\else
  \usepackage[utf8]{inputenc}
\fi

\usepackage[european,americanvoltages]{circuitikz}

\usepackage{url}

% Underfigurer
\renewcommand{\thesubfigure}{(\arabic{subfigure})}

% Overskrift
\emnekode{TMA4101}
\emnenavn{Matematikk 1 for MTELSYS}
\title{Lineæravbildninger}

% Nye kommandoer
\newcommand{\dee}{\mathop{}\!{d}}

\begin{document}
\pagenumbering{gobble}

\maketitle


\begin{tcolorbox}
\end{tcolorbox}

Du har kanskje lagt merke til at matrisevektorproduktet tilfredsstiller regnereglene for lineæravbildninger.

\begin{tcolorbox}[colback=blue!10, title=Superposisjonsprinsippet på matematikermåten]
La $\mathbf{ x}$ og $\mathbf{ y}$ være en vektor og $c$ være en skalar.
En lineæravbildning er en funksjon $T$ slik at 
\begin{align*}
T(\mathbf x+\mathbf y)&=T(\mathbf x)+T(\mathbf y) \\
T(c\mathbf x)&=cT(\mathbf x)
\end{align*}
\end{tcolorbox}


\section*{Standardoppgaver}


\begin{oppgave}{S1}
\end{oppgave}

\section*{Viderekomne oppgaver}



\end{document}
