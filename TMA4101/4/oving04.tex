\documentclass[a4paper,norsk,11pt]{interaktiv}

% Importerte pakker
\usepackage{float}
\restylefloat{figure}
\usepackage{ifluatex}
\usepackage{subfigure}
\usepackage{tikz}
\usetikzlibrary{arrows}
\usepackage[parfill]{parskip}    		% Activate to begin paragraphs with an empty line rather than an indent
\usepackage{graphicx}

\usepackage[american]{circuitikz}
\ifluatex
  \usepackage{fontspec}
  \setmainfont{Calibri}
  \usepackage{unicode-math}
  \setmathfont{Cambria Math}
\else
  \usepackage[utf8]{inputenc}
\fi

% Underfigurer
\renewcommand{\thesubfigure}{(\arabic{subfigure})}

% Overskrift
\emnekode{TMA4101}
\emnenavn{Matematikk 1 for MTELSYS}
\title{Øving 4 - Funksjoner på $\mathbb{R}$}
%\author{Fourieranalyse}


% Nye kommandoer
\newcommand{\dee}{\mathop{}\!{d}}

\begin{document}
\pagenumbering{gobble}

\maketitle



\section*{Obligatoriske oppgaver}


\begin{oppgave}{E1}
I en tidligere øving summerte vi spenningsfallet over en krets, 
og fikk likningen 
\[
2=v+\exp{v}.
\]
Vis at denne likningen har entydig løsning $x \geq 0$.
\end{oppgave}

\begin{oppgave}{E2}
Bestem konstantene $a$, $b$ og $c$ slik at funksjonen
\[
f(x)=
\begin{cases}
0 \quad &x<1 \\
ax^2+bx+c \quad &1\leq x \leq 2\\
1 \quad &x  >2
\end{cases}
\]
blir kontinuerlig i $x=2$ og kontinuerlig deriverbar i $x=1$.
Skisser $f$ og $f'$.
\end{oppgave}




\section*{Anbefalte oppgaver}



\begin{oppgave}{B1}
La $f: [0,\infty)\to \mathbb{R}$ være gitt ved $f(x)=x+\exp{x}$. 
Avgjør om $f^{-1}$ eksisterer, 
og finn $f^{-1}$ dersom dette er mulig.
\end{oppgave}

\begin{oppgave}{A2}
Vis at eksponensialfunksjonen er positiv overalt.
\end{oppgave}

\begin{oppgave}{C3}
Vis at likningen $2=v+\exp{v}$ har entydig løsning $x \in \mathbb{R}$.
(Hint: bruk forrige oppgave.)
\end{oppgave}

\begin{oppgave}{C4}
La $f: \mathbb{R} \to \mathbb{R}$ være gitt ved $f(t)=|2+t^3|$.  Finn ut hvor $f'$ eksisterer, og bestem uttrykket for $f'(x)$. 
\end{oppgave}

\begin{oppgave}{C5}
La $f: \mathbb{R} \to \mathbb{R}$ være gitt ved $f(t)=\frac{1}{1+x^2}$.  Finn ut hvor $f$ er avtagende og hvor $f$ er  økende.
\end{oppgave}


\begin{oppgave}{B6}
Finn tangentlinjen til sirkelen med likning $x^2+y^2=5$ i punktet $x=1$, $y=2$. 
\end{oppgave}


\section*{Relevante eksamensoppgaver fra TMA4100}

\begin{oppgave}{C}
2020K oppgave 3
\end{oppgave}

\begin{oppgave}{B}
2019H oppgave 4
\end{oppgave}

\begin{oppgave}{A}
2019H oppgave 10
\end{oppgave}

\begin{oppgave}{D}
2019K oppgave 1
\end{oppgave}

\begin{oppgave}{B}
2019K oppgave 9
\end{oppgave}

\begin{oppgave}{D}
2018H oppgave 1
\end{oppgave}

\begin{oppgave}{B}
2018H oppgave 5
\end{oppgave}


\begin{oppgave}{D}
2000H oppgave 5
\end{oppgave}

\begin{oppgave}{A}
2000H oppgave 007
\end{oppgave}

\begin{oppgave}{C}
1999H oppgave 2
\end{oppgave}

\begin{oppgave}{C}
1999H oppgave 3a) \\
(Oppgave b) skal vi lære å løse senere i semesteret.)
\end{oppgave}

%\section*{Vanskelige oppgaver}

\end{document}
