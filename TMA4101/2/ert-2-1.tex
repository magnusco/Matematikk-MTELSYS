
\documentclass[a4paper,norsk,11pt]{interaktiv}
\usepackage{tcolorbox}
% Importerte pakker
\usepackage{float}
\restylefloat{figure}
\usepackage{ifluatex}
\usepackage{subfigure}
\usepackage{tikz}
\usetikzlibrary{arrows}
\usepackage[parfill]{parskip}    		% Activate to begin paragraphs with an empty line rather than an indent
\usepackage{graphicx}
\usepackage[standard-baselineskips]{cmbright}

\newcommand{\V}[1]{\mathbf{#1}}
\usepackage{unicode-math}

\ifluatex
  \usepackage{fontspec}
  \setmainfont{Calibri}
  \usepackage{unicode-math}
  \setmathfont{Cambria Math}
\else
  \usepackage[utf8]{inputenc}
\fi

\usepackage[european,americanvoltages]{circuitikz}

\usepackage{url}

\newenvironment{amatrix}[1]{% "augmented matrix"
  \left[\begin{array}{*{#1}{c}|c}
}{%
  \end{array}\right]
}
% Underfigurer
\renewcommand{\thesubfigure}{(\arabic{subfigure})}

% Overskrift
\emnekode{TMA4101}
\emnenavn{Matematikk 1 for MTELSYS}




% Nye kommandoer
\newcommand{\dee}{\mathop{}\!{d}}



\title{Determinanter}


\begin{document}
\pagenumbering{gobble}

\maketitle


I denne øvingen skal vi se på en enkel metode for å avgjøre om lineære likningssystemer har løsning, 
og hvorvidt denne er entydig. 

\section*{Standardoppgaver}


I forrige uke løste vi systemene
\[
\begin{amatrix}{3}
2 & 3 & 4 &4 \\
3 & 4 & 5 &5 \\
4 & 5 & 7 &3
\end{amatrix}
\]
og
\[
\begin{amatrix}{3}
2 & 3 & 4 &4 \\
3 & 4 & 5 &5 \\
4 & 5 & 6 &3
\end{amatrix}
\]
og
\[
\begin{amatrix}{3}
2 & 3 & 4 &4 \\
3 & 4 & 5 &5 \\
4 & 5 & 6 &6
\end{amatrix}
\]
og
\[
\begin{amatrix}{3}
2 & 3 & 4 &0 \\
3 & 4 & 5 &0 \\
4 & 5 & 6 &0
\end{amatrix}
\]

\begin{oppgave}{S1}
Alle punkter $(x,y,z)$ som passer i likningen 
\[
ax+by+cz=d
\]
ligger på ett og samme plan i rommet. 
(Dette lærte du på skolen.)
Bruk dette til å forklare alt som skjedde i systemene ovenfor,
og forklar hvordan du kan se om et når et $3\times 3$-likningssystem har entydig løsning.
\end{oppgave}

Gå nå i en eller annen kilde og finn ut hva en determinant er og hvordan man beregner den.

\begin{oppgave}{S2}
Forklar hvorfor
\[
\begin{amatrix}{4}
3&-1&-1&0& 1 \\
-1&3&0&-1& 0 \\
-1&0&3&-1& 0 \\
0&-1&-1&3& 0 
\end{amatrix}
\]
har entydig løsning.
\end{oppgave}

\begin{oppgave}{S3}
Avgjør om likningssystemet
\[
\begin{amatrix}{4}
3&-1&-1&0& 0 \\
-1&3&0&-1& 0 \\
-1&0&3&-1& 0 \\
1&2&2&1& 0 
\end{amatrix}
\]
har entydig løsning.
\end{oppgave}

\section*{Viderekomne oppgaver}

\begin{oppgave}{V1}
Ta stilling til påstanden:
\begin{tcolorbox}
En lineær resistiv krets gir alltid opphav til en linæert likningssystem med entydig løsning.
\end{tcolorbox}
\end{oppgave}



\begin{oppgave}{V2}
Løs likningssystemet
\[
\begin{amatrix}{2}
a&b&m \\
d&e& n 
\end{amatrix}
\]
\end{oppgave}

\begin{oppgave}{V3}
Løs likningssystemet
\[
\begin{amatrix}{7}
0&1&1&0&0&0&1&0 \\
0&0&0&0&1&0&1&0 \\
0&0&0&0&0&1&1&0 \\
0&0&0&0&0&0&0&0 
\end{amatrix}
\]
\end{oppgave}


\end{document}
