\documentclass[a4paper,norsk,11pt]{interaktiv}
\usepackage{tcolorbox}
% Importerte pakker
\usepackage{float}
\restylefloat{figure}
\usepackage{ifluatex}
\usepackage{subfigure}
\usepackage{tikz}
\usetikzlibrary{arrows}
\usepackage[parfill]{parskip}    		% Activate to begin paragraphs with an empty line rather than an indent
\usepackage{graphicx}
\usepackage[standard-baselineskips]{cmbright}

\newcommand{\V}[1]{\mathbf{#1}}
\usepackage{unicode-math}

\ifluatex
  \usepackage{fontspec}
  \setmainfont{Calibri}
  \usepackage{unicode-math}
  \setmathfont{Cambria Math}
\else
  \usepackage[utf8]{inputenc}
\fi

\usepackage[european,americanvoltages]{circuitikz}

\usepackage{url}

\newenvironment{amatrix}[1]{% "augmented matrix"
  \left[\begin{array}{*{#1}{c}|c}
}{%
  \end{array}\right]
}
% Underfigurer
\renewcommand{\thesubfigure}{(\arabic{subfigure})}

% Overskrift
\emnekode{TMA4101}
\emnenavn{Matematikk 1 for MTELSYS}




% Nye kommandoer
\newcommand{\dee}{\mathop{}\!{d}}


\title{Lineær uavhengighet}

\begin{document}
\pagenumbering{gobble}

\maketitle

I denne øvingen skal vi bli kjent med et av de viktigste begrepene i lineæralgebra.

\begin{tcolorbox}
Vi sier at vektorer $\V v_1$, $\V v_2$ ... $\V v_n$ er lineært uavhengige dersom
\[
c_1\V v_1+c_2\V v_2 +\cdots +c_n\V v_n=\V 0
\]
impliserer 
at
\[
c_1=c_2=\cdots=c_n= 0.
\]
\end{tcolorbox}



\section*{Standardoppgaver}

\begin{oppgave}{S1}
Forklar hvorfor 
\[
x_1
\begin{bmatrix}
1\\
2\\
3\\
4
\end{bmatrix}
+
x_2
\begin{bmatrix}
2\\
3\\
4\\
5
\end{bmatrix}
+
x_3
\begin{bmatrix}
3\\
4\\
5\\
6
\end{bmatrix}
=
\begin{bmatrix}
-3\\
-7\\
-3\\
0
\end{bmatrix}
\]
og 
\[
\begin{amatrix}{3}
1&2&3&-3\\
2&3&4&-7\\
3&4&5&-3\\
4&5&6&0
\end{amatrix}
\]
beskriver de samme likningene.
\end{oppgave}


Vi har så vidt sett at dersom systemet ikke er kvadratisk, 
kan man ikke avgjøre spørsmålet om entydighet ved å beregne determinant. 

\begin{oppgave}{S2}
Løs likningssystemet
\[
\begin{amatrix}{3}
1&2&3&-3\\
2&3&4&-7\\
3&4&5&-3\\
4&5&6&0
\end{amatrix}
\]
\end{oppgave}


\begin{oppgave}{S3}
Løs likningssystemet
\[
\begin{amatrix}{3}
8&-7&0&-3\\
-8&-7&3&-7\\
-4&5&-8&-3\\
-6&6&-4&0
\end{amatrix}
\]
\end{oppgave}

Dersom man skal skjønne forskjellen på de to systemene over, 
er lineær uavhengighet det riktige rammeverket å bruke. 
Dette er mer anvendelig konsept enn determinanter, 
og vi skal få bruk for det mange ganger dette studieåret.

\begin{oppgave}{S4}
Løs likningssystemet
\[
\begin{amatrix}{3}
1&2&3&0\\
2&3&4&0\\
3&4&5&0\\
4&5&6&0
\end{amatrix}
\]
\end{oppgave}


\begin{oppgave}{S5}
Løs likningssystemet
\[
\begin{amatrix}{3}
8&-7&0&0\\
-8&-7&3&0\\
-4&5&-8&0\\
-6&6&-4&0
\end{amatrix}
\]
\end{oppgave}


\section*{Viderekomne oppgaver}


\begin{oppgave}{V1}
Kan en lineært uavhengig vektormengde inneholde nullvektoren?
\end{oppgave}

\begin{oppgave}{V2}
La 
\begin{align*}
p(x)&=8x^3-8x^2-4x-6 \\
q(x)&=-7x^3-7x^2+5x+6 \\
r(x)&=3x^2-8x-4 
\end{align*}
Finnes det konstanter $a$, $b$ og $c$ slik at 
\[
a\cdot p(x)+b\cdot q(x)+c\cdot r(x)=-3x^3-7x^2-3x?
\]
\end{oppgave}

\begin{oppgave}{V3}
La 
\begin{align*}
p(x)&=-6x^3-4x^2-8x+8 \\
q(x)&=6x^3+5x^2-7x-7 \\
r(x)&=-4x^3-8x^2+3x
\end{align*}
Finnes det konstanter $a$, $b$ og $c$ slik at 
\[
a\cdot p(x)+b\cdot q(x)+c\cdot r(x)=-3x^2-7x-3?
\]
\end{oppgave}

\end{document}
