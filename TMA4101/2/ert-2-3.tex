\documentclass[a4paper,norsk,11pt]{interaktiv}
\usepackage{tcolorbox}
% Importerte pakker
\usepackage{float}
\restylefloat{figure}
\usepackage{ifluatex}
\usepackage{subfigure}
\usepackage{tikz}
\usetikzlibrary{arrows}
\usepackage[parfill]{parskip}    		% Activate to begin paragraphs with an empty line rather than an indent
\usepackage{graphicx}
\usepackage[standard-baselineskips]{cmbright}

\newcommand{\V}[1]{\mathbf{#1}}
\usepackage{unicode-math}

\ifluatex
  \usepackage{fontspec}
  \setmainfont{Calibri}
  \usepackage{unicode-math}
  \setmathfont{Cambria Math}
\else
  \usepackage[utf8]{inputenc}
\fi

\usepackage[european,americanvoltages]{circuitikz}

\usepackage{url}

\newenvironment{amatrix}[1]{% "augmented matrix"
  \left[\begin{array}{*{#1}{c}|c}
}{%
  \end{array}\right]
}
% Underfigurer
\renewcommand{\thesubfigure}{(\arabic{subfigure})}

% Overskrift
\emnekode{TMA4101}
\emnenavn{Matematikk 1 for MTELSYS}




% Nye kommandoer
\newcommand{\dee}{\mathop{}\!{d}}

\title{Superposisjonsprinsippet}


\begin{document}
\pagenumbering{gobble}

\maketitle

De enkleste modelleringsproblemene i anvendelser er ofte \textbf{lineære}.



\section*{Standardoppgaver}

Det er viktig å vite hvordan man ganger sammen vektorer og matriser.
Det gjøres slik:
\[
\begin{bmatrix}
2 & 3 & 4 \\
3 & 4 & 5 \\
4 & 5 & 7 \\
1 & 1 & 1
\end{bmatrix}
\begin{bmatrix}
x_1 \\
x_2 \\
x_3
\end{bmatrix}
=
x_1
\begin{bmatrix}
2  \\
3  \\
4 \\
1
\end{bmatrix}
+
x_2
\begin{bmatrix}
 3  \\
4  \\
 5 \\
 1
\end{bmatrix}
+
x_3
\begin{bmatrix}
4 \\
5 \\
7 \\
1
\end{bmatrix}
\]
Det er viktig å forstå at dette er en definisjon. 
Vi ganger sammen matriser og vektorer slik fordi noen har skjønt at det kommer noen godt ut av dette. 
Nå skal vi se litt på hva.


\begin{tcolorbox}[colback=blue!10, title=Superposisjonsprinsippet]
La $\mathbf x$ og $\mathbf y$ være vektorer, 
og $c$ en skalar. 
En lineæravbildning $T$ er en funksjon som tilfredsstiller
\begin{align*}
T(\mathbf x+\mathbf y)&=T(\mathbf x)+T(\mathbf y) \\
T(c\mathbf x)&=cT(\mathbf x)
\end{align*}
\end{tcolorbox}


\begin{oppgave}{S1}
Sett opp et likningssystem for kretsen under (alle motstandene er $1\Omega$),
og løs likningssystemet for
\[
\begin{bmatrix}
V_1 \\
V_2 \\
V_3 \\
V_4 
\end{bmatrix}
=
\begin{bmatrix}
1 \\
0 \\
0 \\
0 
\end{bmatrix},
\quad 
\begin{bmatrix}
V_1 \\
V_2 \\
V_3 \\
V_4 
\end{bmatrix}
=
\begin{bmatrix}
0 \\
1 \\
0 \\
0 
\end{bmatrix},
\quad
\begin{bmatrix}
V_1 \\
V_2 \\
V_3 \\
V_4 
\end{bmatrix}
=
\begin{bmatrix}
0 \\
0 \\
1 \\
0 
\end{bmatrix}
\quad
\text{og}
\quad
\begin{bmatrix}
V_1 \\
V_2 \\
V_3 \\
V_4 
\end{bmatrix}
=
\begin{bmatrix}
0 \\
0 \\
0 \\
1 
\end{bmatrix}
\]
\begin{center}
	\begin{circuitikz}
		\draw  (0,3) to [V=$V_1$] (0,0);
		\draw  (0,3) to [R] (3,3) to [R] (3,0) to [R] (0,0); 
		\draw  (3,3) to [R] (6,3);
		\draw (3,0) to [R] (6,0); 
		\draw  (6,3) to [V=$V_3$] (6,0);
		\draw  (0,6) to [R] (3,6) to [R] (3,3); 
		\draw  (0,6) to [V=$V_2$] (0,3);
		\draw  (3,6) to [R] (6,6); 
		\draw (6,6) to [V=$V_4$] (6,3);
		\end{circuitikz}
\end{center}
\end{oppgave}

\begin{oppgave}{S2}
Finn strømmene i kretsen når
\[
\begin{bmatrix}
V_1 \\
V_2 \\
V_3 \\
V_4 
\end{bmatrix}
=
\begin{bmatrix}
4 \\
3 \\
2 \\
1 
\end{bmatrix}
\]
\end{oppgave}



\section*{Viderekomne oppgaver}

\begin{oppgave}{V1}
Vis at matrise-vektorproduktet er en lineæravbildning, 
og forklar hvorfor dette impliserer at du kan basere din løsning av S2 på din løsning av S1.
\end{oppgave}

\begin{oppgave}{V2}
Finn strømmene i kretsen. (Alle motstandene er fortsatt 1$\Omega$. Her kan python komme godt med, så du slipper å gausseliminere en $8\times 8$-matrise.)
\begin{center}
	\begin{circuitikz}
		\draw  (0,3) to [V=$V_1$] (0,0);
		\draw  (0,3) to [R] (3,3) to [R] (3,0) to [R] (0,0); 
		\draw  (3,3) to [R] (6,3);
		\draw (3,0) to [R] (6,0); 
		\draw  (6,3) to [V=$V_3$] (6,0);
		\draw  (0,6) to [R] (3,6) to [R] (3,3); 
		\draw  (0,6) to [V=$V_2$] (0,3);
		\draw  (3,6) to [R] (6,6); 
		\draw (6,6) to [V=$V_4$] (6,3);
		\draw  (6,3) to [R] (9,3) to [R] (9,0) to [R] (6,0); 
		\draw  (9,3) to [R] (12,3);
		\draw (9,0) to [R] (12,0); 
		\draw  (12,3) to [V=$V_5$] (12,0);
		\draw  (6,6) to [R] (9,6) to [R] (9,3); 
		\draw  (9,6) to [R] (12,6); 
		\draw (12,6) to [V=$V_6$] (12,3);
		\end{circuitikz}
\end{center}
\end{oppgave}

\end{document}
