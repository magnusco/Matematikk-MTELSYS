\documentclass[a4paper,norsk,11pt]{interaktiv}

% Importerte pakker
\usepackage{float}
\restylefloat{figure}
\usepackage{ifluatex}
\usepackage{subfigure}
\usepackage{tikz}
\usetikzlibrary{arrows}
\usepackage[parfill]{parskip}    		% Activate to begin paragraphs with an empty line rather than an indent
\usepackage{graphicx}

\ifluatex
  \usepackage{fontspec}
  \setmainfont{Calibri}
  \usepackage{unicode-math}
  \setmathfont{Cambria Math}
\else
  \usepackage[utf8]{inputenc}
\fi

\usepackage{url}

% Underfigurer
\renewcommand{\thesubfigure}{(\arabic{subfigure})}

% Overskrift
\emnekode{TMA4111}
\emnenavn{Matematikk 3 for MTELSYS}
\title{Lienards likning}



% Nye kommandoer
\newcommand{\dee}{\mathop{}\!{d}}

\begin{document}
\pagenumbering{gobble}

\maketitle


Øvingsopplegget i TMA4111 er organisert i samme ånd som i TTT4203 - hver øving skal gi deg erfaring og trening, 
samt stimulere til refleksjon. 



Denne uken skal vi studere Lienards likning
\[
\ddot{x}+ f(x)\dot{x}+g(x)x=0.
\]


\begin{oppgave}{1}
Forklar hvorfor dette ikke er en lineær likning, 
og skriv om til et førsteordens system.
\end{oppgave}

Van der Pols likning er et spesialtilfelle av Lienards likning, 
og ser slik ut:
\[
\ddot{x}+ \mu(1-x^2)\dot{x}+x=0.
\]
Den ble oppdaget av Balthasar van der Pol (\url{https://en.wikipedia.org/wiki/Balthasar_van_der_Pol}) i forbindelse med arbeid på radiorør,
men kan også modellere andre ting, for eksempel hjerteslag.


\begin{oppgave}{2}
Skriv et script som løser van der Pols likning numerisk med symplektisk Euler. 
\end{oppgave}




\end{document}
