\documentclass[a4paper,norsk,11pt]{interaktiv}

% Importerte pakker
\usepackage{float}
\restylefloat{figure}
\usepackage{ifluatex}
\usepackage{subfigure}
\usepackage{tikz}
\usetikzlibrary{arrows}
\usepackage[parfill]{parskip}    		% Activate to begin paragraphs with an empty line rather than an indent
\usepackage{graphicx}

\ifluatex
  \usepackage{fontspec}
  \setmainfont{Calibri}
  \usepackage{unicode-math}
  \setmathfont{Cambria Math}
\else
  \usepackage[utf8]{inputenc}
\fi

\usepackage{url}

% Underfigurer
\renewcommand{\thesubfigure}{(\arabic{subfigure})}

% Overskrift
\emnekode{TMA4111}
\emnenavn{Matematikk 3 for MTELSYS}
\title{Fouriertransform}



% Nye kommandoer
\newcommand{\dee}{\mathop{}\!{d}}

\begin{document}
\pagenumbering{gobble}

\maketitle

Det går an å vise at dersom $x$ er et signal som oppfører seg pent (ikke tenk så mye på hva det vil si inntil videre), 
og
\[
X(\omega) = \mathcal{F}\left\{ x(t) \right\}=\int_{-\infty}^{\infty} x(t)e^{-i\omega t}\;dt
\]
er
\[
x(t) = \frac{1}{2\pi}\int_{-\infty}^{\infty}X(\omega)e^{i\omega t}d\omega.
\]
Dette kalles fouriertransform, 
og er ganske vanskelig å skjønne noe av i begynnelsen. 
Men fouriertransform er helt ekstremt viktig i både matematikk og anvendelser. 
Det kan brukes til alt: \url{https://en.wikipedia.org/wiki/Fourier_transform}.

\begin{oppgave}{1}
Finn fouriertransformen til 
\[
f(t)=
\begin{cases} 
1 \quad |t|<1 \\
0 \quad |t|\geq 1
\end{cases}
\]
\end{oppgave}

\begin{oppgave}{2}
Finn fouriertransformen til 
\[
f(t)=
\begin{cases} 
a \quad |t|<1/a \\
0 \quad |t|\geq 1/a
\end{cases}
\]
\end{oppgave}


\begin{oppgave}{3}
Finn fouriertransformen til $f(t)=e^{-t^2}$.
\end{oppgave}

\end{document}
