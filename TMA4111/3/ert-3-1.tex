\title{Fouriertransform}

\documentclass[a4paper,norsk,11pt]{interaktiv}
\usepackage{tcolorbox}
% Importerte pakker
\usepackage{float}
\restylefloat{figure}
\usepackage{ifluatex}
\usepackage{subfigure}
\usepackage{tikz}
\usetikzlibrary{arrows}
\usepackage[parfill]{parskip}    		% Activate to begin paragraphs with an empty line rather than an indent
\usepackage{graphicx}
\usepackage[standard-baselineskips]{cmbright}

\newcommand{\V}[1]{\mathbf{#1}}
\usepackage{unicode-math}

\ifluatex
  \usepackage{fontspec}
  \setmainfont{Calibri}
  \usepackage{unicode-math}
  \setmathfont{Cambria Math}
\else
  \usepackage[utf8]{inputenc}
\fi

\usepackage[european,americanvoltages]{circuitikz}

\usepackage{url}

\newenvironment{amatrix}[1]{% "augmented matrix"
  \left[\begin{array}{*{#1}{c}|c}
}{%
  \end{array}\right]
}
% Underfigurer
\renewcommand{\thesubfigure}{(\arabic{subfigure})}

% Overskrift
\emnekode{TMA4101}
\emnenavn{Matematikk 1 for MTELSYS}




% Nye kommandoer
\newcommand{\dee}{\mathop{}\!{d}}





\begin{document}
\pagenumbering{gobble}

\maketitle




Det går an å vise at dersom $x=x(t)$ er et signal som oppfører seg pent (ikke tenk så mye på hva det vil si inntil videre), 
og
\[
X(\omega) = \mathcal{F}\left\{ x \right\}=\int_{-\infty}^{\infty} x(t)e^{-i\omega t}\;dt
\]
er
\[
x(t) = \frac{1}{2\pi}\int_{-\infty}^{\infty}X(\omega)e^{i\omega t}d\omega.
\]
Dette kalles fouriertransform, 
og er ganske vanskelig å skjønne noe av i begynnelsen. 
Men fouriertransform er helt ekstremt viktig i både matematikk og anvendelser. 
Det kan brukes til alt: \url{https://en.wikipedia.org/wiki/Fourier_transform}.
I et klassisk matematikkemne brukes fouriertransform først og fremst til å løse partielle differensiallikninger.

Vi begynner med å beregne et par transformer. 

\begin{oppgave}{1}
Finn fouriertransformen til 
\[
f(t)=
\begin{cases} 
1 \quad |t|<1 \\
0 \quad |t|\geq 1
\end{cases}
\]
\end{oppgave}

\begin{oppgave}{2}
Finn fouriertransformen til 
\[
f(t)=
\begin{cases} 
a \quad |t|<1/a \\
0 \quad |t|\geq 1/a
\end{cases}
\]
Hva skjer med fouriertransformen når $a\to \infty$?
\end{oppgave}

\begin{oppgave}{3}
Finn fouriertransformen til
\[
f(t)=e^{-a|t|}
\]
\end{oppgave}

I de to neste oppgavene må du nesten anta at inversformelen gjelder, og bruke de foregående oppgavene.

\begin{oppgave}{4}
Finn fouriertransformen til 
\[
f(t)=\frac{\sin t}{t}
\]
\end{oppgave}

\begin{oppgave}{5}
Finn fouriertransformen til
\[
f(t)=\frac{1}{t^2+a^2}
\]
\end{oppgave}


Fouriertransformen er en lineæroperator: 
\[
\mathcal{F}\left\{ x+y \right\}=\int_{-\infty}^{\infty} \left(x(t)+y(t)\right)e^{-i\omega t}\;dt=\int_{-\infty}^{\infty} x(t)e^{-i\omega t}\;dt+=\int_{-\infty}^{\infty} y(t)e^{-i\omega t}\;dt=X(\omega) +Y(\omega) 
\]
Det er greit nok. 
En mindre innlysende, 
men veldig viktig regneregel er:

\begin{tcolorbox}
\[
\mathcal{F}(f')=iw\mathcal{F}(f)
\]
\end{tcolorbox}

\begin{oppgave}{6}
Vis dette (hint: delvis integrasjon). Hva må du anta om $x$ for at formelen skal gjelde? 
\end{oppgave}

%Her er en til:
%
%\begin{tcolorbox}
%\[
%\mathcal{F}(x(at))=\frac{1}{a}X\left(\frac{\omega}{a}\right).
%\]
%\end{tcolorbox}

%\begin{oppgave}{6}
%Vis dette.
%\end{oppgave}


\end{document}
